Schema.org\\
Luego de los fragmentos enriquecidos y el improvisado vocabulario data-vocabulary creado para soportar los fragmentos, Google 
en conjunto con Bing y Yahoo crearon en el 2011 Schema.org, con el objetivo de obtener un vocabulario más completo y 
organizado para la implementación de los snippets con RDFa y microdata.\\
Su lanzamiento provocó la inmediata obsoletización de data-vocabulary cuyas clases fueron todas reemplazadas por equivalentes 
dentro de la nueva ontología:\\
http://www.data-vocabulary.org/Address -> http://schema.org/PostalAddress\\
http://www.data-vocabulary.org/Geo -> http://schema.org/GeoCoordinates\\
http://www.data-vocabulary.org/Organization -> http://schema.org/Organization\\
http://www.data-vocabulary.org/Person -> http://schema.org/Person\\
http://www.data-vocabulary.org/Event -> http://schema.org/Event\\
http://www.data-vocabulary.org/Product -> http://schema.org/Product\\
http://www.data-vocabulary.org/Review -> http://schema.org/Review\\
http://www.data-vocabulary.org/Offer -> http://schema.org/Offer\\
Este nuevo vocabulario, recibe constantes actualizaciones, al punto que, al día de la fecha, la ontología cuenta con 946 clases. \\
En particular, la clase review, definida bajo el namespace http://schema.org/Review y es subclase de schema:CreativeWork 
que a su vez es subclase de schema:Thing .\\
Propiedades de Review\\
itemReviewed	El ítem que está siendo evaluado, tiene rango schema:Thing \\
reviewBody 	El valor textual del review de la evaluación, que tiene rango schema:Text \\
reviewRating 	El valor numérico del review de la evaluación, que tiene rango schema:Rating \\
\\Propiedades de CreativeWork \\
about 	El tema del contenido. Rango schema:Thing. \\
aggregateRating 	El promedio de la acumulación de uno o más ratings dentro de un review. Tiene rango schema:AggregateRating \\
author 	El autor del contenido. Tiene rango schema:Person o schema:Organization\\
comment 	Comentarios sobre el contenido. De rango schema:Comment o schema:UserComments.\\
commentCount 	Cantidad de comentarios. Rango schema:Integer . \\
creator 	El autor del contenido. Tiene rango schema:Person o schema:Organization\\
dateCreated 	Fecha en la que fue creado. Rango schema:Date .\\
dateModified 	Fecha en la que fue modificado por última vez. Rango schema:Date .\\
datePublished 	Fecha en de la primer publicación. Rango schema:Date .\\
publisher 	El publicador. Tiene rango schema:Organization\\
review 	 	El review sobre el contenido. Tiene rango schema:Review .\\
text 		Contenido textual. Tiene rango schema:Text .\\
\\Properties de Thing\\
additionalType 	 Tipos adicionales para el ítem que en general son utilizados para especificar tipos externos, como podría ser por ejemplo una película de clase http://schema.org/Movie con el tipo adicional  http://dbpedia.org/ontology/ . Tiene rango schema:URL .\\
alternateName 	 Especifica un alias. Rango schema:Text .\\
description 	Una descripción corta del ítem. Rango schema:Text \\
name 		Establece el nombre. Rango schema:Text .\\
sameAs 		Una referencia a una url que desambiguadamente represente al ítem, como podría ser una URL de wikipedia, dbpedia, freebase, etc . Rango schema:URL.\\
url 		URL del ítem. Rango schema:URL\\
\\
Unas 60 propiedades de CreativeWork fueron omitidas por no iban al caso ya que sólo tienen razón de ser para otras clases que 
también heredan de CreativeWork. Lo importante aquí es remarcar un par de situaciones interesantes:\\
=Las propiedades Review:reviewBody, CreativeWork:text y Thing:description podrían contener el valor textual de la evaluación del review semánticamente correcta por la definición de cada una.\\
=Las propiedades CreativeWork:author y CreativeWork:creator son exactamente iguales. \\
=Los valores CreativeWork:dateCreated CreativeWork:dateModified y CreativeWork:datePublished podrían generar confusión. \\
=Sería sintácticamente correcto que un Review utilice la propiedad Review .\\
\\
Más adelante se mostrarán y enumerarán los problemas que estas cuestiones (causadas por el afán de hacer uso de la herencia
lo más posible)  generan. \\