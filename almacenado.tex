\chapter{Extracción y Almacenamiento de los Datos}
\label{chapter:extraccion}

En la sección \ref{section:extraccion} se plantearon los requisitos y opciones que se requieren
para el tratamiento de la información obtenida en el paso anterior.

\section{Objetivo}

\noindent Convertir los documentos con información semántica embebida en documentos HTML en documentos RDF, que luego puedan ser 
almacenados en un triplestore y almacenarlos.
\\\\
La necesidad de tener la información en un triplestore surge de varios puntos:
\begin{enumerate}
 \item Tener la información centralizada, así, por cada paso siguiente a realizar, resulte mucho más simple aplicar un mismo proceso a todos los datos.
 \item Tener la información en un mismo lenguaje, por la misma razón que el punto anterior.
 \item Poder realizar queries SPARQL tanto para generar estadísticas como para realizar un curado de la información.
 \item Poder utilizar el motor de inferencias para detectar errores en los documentos.
\end{enumerate}

\section{Estrategia}

\noindent El extractor utilizado fue any23 con su librería para java. El motivo es que es el único de los extractores que provee 
librería para java, y además soporta todos los lenguajes de RDF embebido en HTML.
\\\\
Los documentos se guardaron en formato n-Quads ya que son mucho más eficientemente procesables y además se puede conservar el grafo 
del cual provienen.
\\\\
Como siguiente paso, con el objetivo de tener un backup ligero de los datos que pueda ser levantado a una base de datos fácilmente, se realizó un merge de todos los documentos extraídos en uno solo.
Para ello utilizó la herramienta RIOT (disponible en la librería any23).
\\\\
Y por último se determinó utilizar una base de datos TDB, por múltiples razones:
\begin{itemize}
\item Es gratuita y opensource. 
\item Es muy eficiente, ya que no trabaja sobre una base de datos MySQL.
\item Se encuentra incorporado a jena por lo que se dispone de su utilización en Java.
\item Para levantar la base de datos, a partir de ese documento nq se utilizó la operación bulkloader2 que la manera más rápida. Con lo que se obtuvo la base de datos TDB.
\end{itemize}

\section{Resultados}

\textbf{Any23}
\\\\
En el cuadro \ref{table:Extractors} puede observarse una lista de los principales extractores utilizados por Any23 para realizar el 
procedimiento. En la columna Extractor se describe el nombre del extractor que fue utilizado, el nombre del extractor identifica también qué formato de 
información extrae, la columna Documentos indica la cantidad de documentos que el extractor procesó (esto implica que el documento contenía el tipo de formato descripto en el nombre del extractor), y 
Porcentaje se refiere al porcentaje de documentos procesados con respecto al total.
\begin{table}[h]
\begin{tabular}{| l | c | r | }\hline
Extractor & Documentos & Porcentaje \\\hline
html-head-title & 233.743 & 98,77 \\    
html-rdfa11 & 156.096 & 65,96 \\
html-microdata & 110.325 & 46,62 \\    
html-mf-hreview & 109.096 & 46,10 \\
html-hcard & 73.197 & 30,93  \\
html-hreview-aggregated & 48.771 & 20,60 \\
html-mf-adf & 44.788 & 18,92 \\\hline
 \end{tabular}
\caption{Extractores utilizados más importantes}
\label{table:Extractors}
\end{table}
\\
\\
Errores por documentos mal descargados:
\\\\
null 4661

\noindent invalid property name " 95

\noindent Error while retrieving mime type 1
\\ 
\\ 
El cuadro \ref{table:SemanticaEnDocumentows} define un panorama de las tripletas semánticas extraídas en los documentos obtenidos.

\begin{table}[h]
\begin{tabular}{| l | c | }\hline
Documentos con review & 187.088 \\
Documentos sin review & 49.552\\
Documentos sin tripletas & 2.878\\
Tripletas totales & 16.614.727\\
Tripletas promedio por documento & 71.66\\
Tripletas promedio en documentos con tripletas & 72.56\\
Tripletas promedio sobre documentos con review & 84.02\\
Tripletas totales de documentos con review & 15.719.392\\
Tripletas promedio sobre documentos sin review & 20.01\\\hline
Total Documentos & 236.640\\\hline
\end{tabular}
\caption{Estadística del contenido semántico en los documentos}
\label{table:SemanticaEnDocumentows}
\end{table}

Cabe destacar que el porcentaje de documentos con Review es 79\% e implica que el 21\% restante fueron documentos que Sindice tiene indexados como contenedores de 
alguna de las clases de review mencionadas lo cual ya no es cierto. Ya sea porque:
\begin{itemize}
\item El documento no existe más
\item El documento cambió y no habla sobre reviews
\item El documento dejó de utilizar la web semántica para publicar reviews
\end{itemize}
Pero ese porcentaje no refleja una estadística sobre un muestreo aleatorio válido de Sindice ya que, como se mencionó anteriormente, los dominios fueron seleccionados 
con un previo checkeo superficial de la información de Sindice sobre cada dominio.
\\
\\
\textbf{Riot:}

\noindent Múltiples errores detectados en el merge de los documentos extraídos. No queda claro si causados por any23 en la extracción, o el documento ya de origen 
mal confeccionado. 

El cuadro \ref{table:BulkLoadErrors} muestra los errores detectados en el proceso de bulk load que no fueron advertidos por any23.\\
\begin{table}[h]
\begin{tabular}{| l | l | }\hline
Error & Razón \\\hline
PORT\_SHOULD\_NOT\_BE\_& Ports under 1024 should be accessed \\ WELL\_KNOWN  in PORT  & using the appropieate scheme name. \\\hline
ILLEGAL\_CHARACTER & The character violates the \\  in FRAGMENT & grammar rules for URIs/IRIs.\\\hline
DEFAULT\_PORT\_SHOULD\_& If the port is the default one \\ BE\_OMITTED in PORT  & for the scheme it should be omitted. \\\hline
REQUIRED\_COMPONENT\_& A component that is required by \\ MISSING in HOST & the scheme is missing.\\\hline
DOUBLE\_WHITESPACE& Either two or more consecutive \\  in QUERY  &  whitespace characters, or leading or \\ & trailing whitespace.These match no \\ & grammar rules  of URIs/IRIs. These \\ &  characters are  permitted in RDF URI \\ &  References, XML  system identifiers,\\ & but not XML Schema  anyURIs.\\\hline
WHITESPACE in PATH & A single whitespace character. These \\ &  match no grammar rules of URIs/IRIs. \\ &  These characters are permitted in \\ & RDF URI References, XML system \\ & identifiers, and XML Schema \\ & anyURIs.\\\hline
ILLEGAL\_PERCENT\_& The host component a percent occurred \\ ENCODING in QUERY  & without two following hexadecimal \\ & digits.\\\hline
NOT\_DNS\_NAME& The host component did not meet \\  in HOST  & the restrictions on DNS names.\\\hline
DNS\_LABEL\_DASH\_START& A DNS name had a - at the beginning \\ \_OR\_END in HOST  & or end.\\\hline
PROHIBITED\_COMPONENT\_& A component that is prohibited by the \\ PRESENT in USER  & scheme is present.\\\hline
WHITESPACE in FRAGMENT & A single whitespace character. These \\ & match no grammar  rules of URIs/IRIs. \\ & These characters are permitted in RDF \\ & URI References, XML system \\ & identifiers, and XML Schema anyURIs.\\\hline
WHITESPACE in QUERY &A single whitespace character. These \\ & match no grammar  rules of URIs/IRIs. \\ & These characters are permitted in RDF \\ & URI References, XML system \\ & identifiers, and XML Schema anyURIs.\\\hline
\end{tabular}
\caption{Errores en el proceso de bulk load}
\label{table:BulkLoadErrors}
\end{table}

\subsection{TDB Bulk load}

La base de datos resultante contiene:
\\\\
Cantidad de grafos: 182153

\noindent Cantidad de grafos que contienen la clase schema:Review: 50955 

\noindent Cantidad de grafos que contienen la clase schema:Aggregate rating: 52022

\noindent Cantidad de grafos que contienen la clase purl:Review: 104316 

\noindent Cantidad de grafos que contienen la clase purl:ReviewAggregate: 43991
\\\\
La búsqueda en sindice se realizó para schema:Review y purl:Review. Sin embargo, como se puede apreciar se obtuvieron muchos resultados con reviews agregados.