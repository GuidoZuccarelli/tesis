\chapter{Extracción y Almacenamiento de los Datos}
\label{chapter:extraccion}

Objetivo

Convertir los documentos con información semántica embebida en documentos HTML en documentos RDF, que luego puedan ser 
almacenados en un triplestore y almacenarlos.

La necesidad de tener la información en un triplestore surge de varios puntos:

Tener la información centralizada, así, por cada paso siguiente a realizar, resulte mucho más simple aplicar un mismo proceso a todos los datos.

Tener la información en un mismo lenguaje, por la misma razón que el punto anterior.

Poder realizar queries SPARQL tanto para generar estadísticas como para realizar un curado de la información.

Poder utilizar el motor de inferencias para detectar errores en los documentos.

Estrategia

El extractor utilizado fue any23 con su librería para java. El motivo es que es el único de los extractores que provee librería para java, y además soporta todos los lenguajes de RDF embebido en HTML.
Los documentos se guardaron en formato n-Quads ya que son mucho más eficientemente procesables y además se puede conservar el grafo del cual provienen.

Como siguiente paso, con el objetivo de tener un backup ligero de los datos que pueda ser levantado a una base de datos fácilmente, se realizó un merge de todos los documentos extraídos en uno solo.
Para ello utilizó la herramienta RIOT (disponible en la librería any23).

Y por último se determinó utilizar una base de datos TDB, por múltiples razones:

Es gratuita y opensource. 

Es muy eficiente, ya que no trabaja sobre una base de datos MySQL.

Se encuentra incorporado a jena por lo que se dispone de su utilización en Java.

Para levantar la base de datos, a partir de ese documento nq se utilizó la operación bulkloader2 que la manera más rapida. Con lo que se obtuvo la base de datos TDB.
\\
\\
Resultados

Any23

Extractores utilizados más importantes:
\begin{tabular}{| l | c | r | }
Extractor & Documentos & Porcentaje \\
html-head-title & 233.743 & 98,77 \\    
html-rdfa11 & 156.096 & 65,96 \\
html-microdata & 110.325 & 46,62 \\    
html-mf-hreview & 109.096 & 46,10 \\
html-hcard & 73.197 & 30,93  \\
html-hreview-aggregated & 48.771 & 20,60 \\
html-mf-adf & 44.788 & 18,92 \\
 \end{tabular}
\\
\\
Errores por documentos mal descargados:

null 4661

invalid property name " 95

Error while retrieving mime type 1
\\ 
 
Documentos obtenidos:

\begin{tabular}{| l | c | }
Documentos con review & 187.088 \\
Documentos sin review & 49.552\\
Documentos sin tripletas & 2.878\\
Tripletas totales & 16.614.727\\
Tripletas promedio por documento & 71.66\\
Tripletas promedio en documentos con tripletas & 72.56\\
Tripletas promedio sobre documentos con review & 84.02\\
Tripletas totales de documentos con review & 15719392\\
Tripletas promedio sobre documentos sin review & 20.01\\
Total Documentos & 236640\\
\end{tabular}

Cabe destacar que, ese procentaje de documentos con review 79\% implica que el 21\% restante fueron documentos que sindice tiene indexados como contenedores de 
alguna de las clases de review mencionadas y ya no los tiene en la actualidad. Ya sea porque:

El documento no existe más

El documento cambió y no habla sobre reviews

El documento dejó de utilizar la web semántica para publicar reviews

Pero ese porcentaje no refleja una estadística sobre un muestreo válido de sindice ya que, com ose mencionó anteriormente, los dominios fueron seleccionados 
dependiendo de si aún existen y además aún tenían publicados documentos de reviews en la web semántica.
\\
\\
Riot:

Múltiples errores detectados en el merge de los documentos extraídos. No queda claro si causados por any23 en la extracción, o el documento ya de origen 
mal confeccionado. Fueron errores  no detectados por any23:

PORT\_SHOULD\_NOT\_BE\_WELL\_KNOWN in PORT: Ports under 1024 should be accessed using the appropieate scheme name. 

ILLEGAL\_CHARACTER in FRAGMENT: The character violates the grammar rules for URIs/IRIs.

DEFAULT\_PORT\_SHOULD\_BE\_OMITTED in PORT: If the port is the default one for the scheme it should be omitted. 

REQUIRED\_COMPONENT\_MISSING in HOST: A component that is required by the scheme is missing.

DOUBLE\_WHITESPACE in QUERY: Either two or more consecutive whitespace characters, or leading or trailing whitespace. These match no grammar rules of URIs/IRIs. These characters are permitted in RDF URI References, XML system identifiers, but not XML Schema anyURIs.

WHITESPACE in PATH: A single whitespace character. These match no grammar rules of URIs/IRIs. These characters are permitted in RDF URI References, XML system identifiers, and XML Schema anyURIs.

ILLEGAL\_PERCENT\_ENCODING in QUERY: The host component a percent occurred without two following hexadecimal digits.

NOT\_DNS\_NAME in HOST: The host component did not meet the restrictions on DNS names.

DNS\_LABEL\_DASH\_START\_OR\_END in HOST: A DNS name had a - at the beginning or end.

PROHIBITED\_COMPONENT\_PRESENT in USER: A component that is prohibited by the scheme is present.

WHITESPACE in FRAGMENT: A single whitespace character. These match no grammar rules of URIs/IRIs. These characters are permitted in RDF URI References, XML system identifiers, and XML Schema anyURIs.

WHITESPACE in QUERY: A single whitespace character. These match no grammar rules of URIs/IRIs. These characters are permitted in RDF URI References, XML system identifiers, and XML Schema anyURIs.
\\
\\
TDB Bulk load

La base de datos resultante contiene:

Cantidad de grafos: 182153

Cantidad de grafos con schema/review: 50955 

Cantidad de grafos con schema/aggregate rating: 52022

Cantidad de grafos con purl/review: 104316 

Cantidad de grafos con purl/aggregate review: 43991

La búsqueda en sindice se realizó para schema/review y purl/review. Sin embargo, como se puede apreciar se obtuvieron muchos resultados con reviews agregados.
