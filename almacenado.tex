\chapter{Extracción y Almacenamiento de los Datos}
\label{chapter:extraccion}

Objetivo

Se realizará un proceso de extracción que convierta los documentos con información semántica embebida en documentos HTML en documentos RDF, que luego puedan ser 
almacenados en un triplestore.

La necesidad de tener la información en un triplestore surge de varios puntos:

Tener la información centralizada, así, por cada paso siguiente a realizar, resulte mucho más simple aplicar un mismo proceso a todos los datos.

Tener la información en un mismo lenguaje, por la misma razón que el punto anterior.

Poder realizar queries SPARQL tanto para generar estadísticas como para realizar un curado de la información.

Poder utilizar el motor de inferencias para detectar errores en los documentos.

Estrategia

Se utilizó any23 para extraer el contenido semántico  y generar por cada documento html, un documento en N-Quads.
El motivo de utilizar any23 es que es la única librería que puede ser utilziada en java, y además soporta todos los lenguajes de RDF embebido en HTML.


Luego se utilizó la herramienta RIOT para hacer un merge de todos los documentos nq y generar uno único.

Y po rúltimo se utilizó la herramienta bulkloader para crear una base de datos TDB a partir del archivo nq que contenía todos los documentos nq.
