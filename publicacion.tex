\chapter{Publicación del Dataset Curado}
\label{chapter:publicacion}

Para este paso, se escogió en un primer lugar el aprovicionamineto de un servicio web para ser accedido públicamente. Esto beneficia 
a los usuarios que requieran hacer un uso inmediato y simple de la información contenida. Ya que si se decidiera proporcionar un 
volcado de datos, el usuario deberá tener conocimiento de la manipulación del disco, así como recursos de hardware y tiempo de configuración.\\
\\
Para proceder con la publicación del dataset de la forma mencionada, el motor que se optó por utilizar es Fuseki, ya que proporciona un
medio de implementación extremadamente sencillo, debido a que funciona perfectamente en conjunto de un dataset TDB, que es con el que 
actualmente se cuenta. \\
Fuseki además, tiene la ventaja de ser fácilmente configurable, y no requiere instalación. En el caso de haber elegido otros motores, 
habría que haber hecho una conversión del dataset para obtenerlo en un formato adecuado, además del esfuerzo necesario para instalar y configurar 
toda la aplicación.\\
\\
Por otro lado se consideró también que podría existir la necesidad por parte de algunos usuarios (la minoría), en hacer un uso más complejo del 
dataset (recordar que la utilización de endpoints sparql tiene sus restricciones), para lo cual requerirían manipular el dataset completo 
por su cuenta.\\
Resolver esta situación, no representó casi inconveniente, debido a la alta capacidad de compresión de datos que posee una base de datos TDB, 
que pasó de 8,4Gb a 1,0Gb. Dicho dataset con esas características puede ser fácilmente publicado para una posible descarga muy fácilmente. 