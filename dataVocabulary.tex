RDF Data Vocabulary\\
En Mayo de 2009 Google anuncia la introdución de los llamados ``Google Rich Snippets'', estos fragmentos enriquecidos son una convención 
de etiquetas (con soporte para RDFa-lite y microdata) que permitían agregar información útil a los SERP del buscador de Google. De manera tal 
que los datos que contenían estos fragmentos, recibían un tratamento especial.\\
Sin embargo en el anuncio, Google revela que el soporte se limitó al uso de las clases y propiedades del vocabulario definido en una página 
notoriamente improvisada llamada http://rdf.datavocabulary.org/ . \\
En ella se establecían modelos de clases para varios tipos de ítems, tales como Persona, Organización o Producto y también para los Reviews.\\
La clase Review, quedó definida bajo el namespace http://data-vocabulary.org/Review e incluía las siguientes propiedades:\\
itemreviewed: Enlace al ítem que está siendo evaluado.\\
\\
rating: El valor numérico con el cual el usuario expresa su satisfacción con el ítem, tiene como rango un valor numérico bajo la clase 
xsd:string o Rating, y los valores posibles se encuentran en escala de 1 a 5, pudiendo la misma ser alterada con la presencia de las 
propiedades worst y best que restringen el valor mínimo y máximo respectivamente.\\
\\
reviewer: El autor del review, su rango es dvocab:Person o xsd:string.\\
\\
dtreviewed: La fecha en la que se realizó el review, no contiene un rango específico pero aclara que debe respetar el formato 
ISO para las fechas.\\
\\
description: El cuerpo del review que representa el valor textual de satisfacción del usuario con el ítem.\\
\\
summary: Un resumen corto del review.\\
\\
Se puede notar la excesiva similitud de este vocabulario con hReview, queda claro que no hubo una intención de innovar algo, 
sino de representar el hReview en microdatos, probablemente por el apuro en la que data vocabulary fue creado.\\
Vale aclarar que limitar las clases y propiedades posibles en RDFa es básicamente hacerle perder el sentido al lenguaje 
(la decentralización de los vocabularios sobre los términos) haciendo que el lenguaje se utilice como si fuese microformatos, 
pero perdiendo su valor más improtante (la simplicidad) de manera tal que tomó la inflexibilidad de microformatos y la complejidad 
de RDFa.\\
Más adelante los Google Rich Snippets incluyeron también soporte para microformatos (lo que incluía hReview). \\