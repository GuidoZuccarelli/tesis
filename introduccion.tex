\chapter{Introducción}
\label{chapter:introduccion}

\section{Evaluaciones entrecruzadas}
\label{section:evaluaciones-cruzadas}
Definiremos a una reseña (de aquí en adelante review) como la forma que tiene un usuario de transmitir mediante el uso de la computadora, su grado de satisfacción con respecto a un ítem. 

Estará conformado mediante una serie de atributos mínimos necesarios y otros que que si bien no serán indispensables, enriquecerán no sólo su valor intrínseco, sino también su utilidad para el contexto que será planteado.


%Item definition
Llamaremos ítem, a un elemento que es aprovechado por las personas, dicho ítem puede ser, un objeto, un servicio, una idea, un programa, etc.  Por ejemplo, una película.

Este último debería poder ser identificable, para ello, será necesario como mínimo un atributo (o un conjunto) que cumpla esa función,  si tomamos como ejemplo un libro, sólo disponer del atributo ISBN, será suficiente para identificarlo.


%Review definition
El review es una expresión de la persona que utilizó el ítem que refleja su grado de conformidad con el mismo, con el objetivo de informarlo a otras personas. Por lo que entonces, el atributo mínimo será aquel que refleje este sentimiento del usuario hacia el ítem, el mismo puede ser un texto explicativo (por ejemplo ``Una película hermosa, pero me parecieron flojos los actores'') o un valor numérico dentro de un rango determinado (7 en escala de 1 a 10).
A partir del atributo base, existen muchos otros atributos que, bien construidos, aportan mucha riqueza al aprovechamiento del review. Por ejemplo, el autor, que identifica a la persona creadora del review. Otro caso es el de la fecha.


\section{La web semántica}
\label{section:la-web-semantica}

% laWebSemantica
En sus comienzos en los 90, la web podía verse como un conjunto de sitios web que ofrecían una colección de documentos web, con el objetivo de comunicar información a los usuarios.

Con el correr de los años, múltiples tecnologías se fueron implementando y permitieron el desarrollo de una web mucho más grande y aprovechable

\section{RDF y OWL} 
\label{section:rdf-owl}

\section{Reviews en la web semántica}
\label{section:reviews-en-la-web}

\section{Linked data cloud}
\label{section:linked-data-cloud}
