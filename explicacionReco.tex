Como se mencionó antes, la web contiene grandes cantidades de documentos publicados con información semántica. Pero la tarea
de encontrarlos, con el agregado de que sólo una pequeña porción de ellos será relevante para los requerimientos no es trivial
en lo absoluto debido a la inmensidad del universo en el que se encuentran. La forma de llevar a cabo este objetivo está 
atada al hardware disponible, tanto para almacenar los datos, como para el tiempo que va a emplear la ejecución de esta 
tarea. \\
Dado que las bases de datos semánticas sólo almacenan información en forma de tripletas o cuadrupletas, los documentos encontrados 
deberán someterse a un proceso de extracción que seleccione las sentencias HTML y las convierta a alguno de los lenguajes que soportan  
tripletas o cuadrupletas. Para esto existen múltiples herramientas.\\
Una vez encontrados, descargados y transformados los documentos HTML a documentos semánticos puede construirse la base de datos semántica 
con la información recolectada.\\
Estos son los cuatro pasos necesarios para lograr tener el dataset semántico con el cual se puede empezar a trabajar. Cada uno de ellos 
posee distintas alternativas para su realización, algunas se describirán a continuación\\