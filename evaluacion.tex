\chapter{Evaluación de Calidad de los Datos}
\label{chapter:evaluacion}

Aquí se realizarán las evaluaciones derivadas de la descripción de la Sección \ref{section:evaluacion}, 
donde se describen los análisis de errores sintácticos y semánticos.

\section{Evaluación nº 1 - Vocabularios}
\label{section:evaluacion-vocabularios}

\subsection*{Objetivo}
Buscar aquellas propiedades y clases ligadas a recursos relevantes (que estén relacionados con algún review mediante property paths) 
para las cuales no se encuentre representada en ninguna ontología. Y luego verificar que genere algún problema para la aplicación a construir.

\subsection*{Estrategia}
Este primer paso es bastante simple, básicamente se deben conseguir todos los vocabularios utilizados (para saber cuales son estos basta con revisar el namespace 
de los prefijos) y cargarlos en el dataset.
\\\\
Una vez con los vocabularios cargados se puede realizar una consulta en SPARQL para las propiedades y luego otra para las clases:
\newpage
\begin{lstlisting}[frame=single]
SELECT ?c (count (distinct ?o) as ?cantidad)
WHERE{
{
?s <http://local.org/id> ?id .
?s (:|!:)+ ?o .
}UNION{
?s <http://local.org/itemId> ?id .
?s (:|!:)+ ?o .
}
?o a ?c .
FILTER NOT EXISTS{
?c a rdfs:Class.
}
}GROUP BY ?c ORDER BY DESC(?cantidad)
\end{lstlisting}

\noindent Luego se puede hacer una consulta igual pero buscando por propiedades.

\subsection*{Resultados}
Los resultados de esta evaluación fueron interesantes. Además de algunas propiedades y clases mal escritas, ya sea por una letra faltante o una minúscula en lugar 
de una mayúscula, se encontró que todas las propiedades utilizadas de la ontología schema estaban incorrectas.
\\\\
Ninguna de las propiedades encontradas bajo el namespace http://schema.org/ pertenecían a la ontología schema.
Esto se debe a que cada propiedad incluía bajo su path su clase dominio de la siguiente manera: \\\noindent http://schema.org/CLASE/Propiedad .
Por ejemplo: \\\noindent
http://schema.org/Product/aggregateRating en lugar de \\\noindent http://schema.org/aggregateRating

Situaciones como ésta generan problemas cuando se intenta realizar consultas generales. Si por ejemplo se requiriera encontrar todos los individuos de la clase schema:aggregateRating no alcanzaría con 
una simple querie, habría que realizar tantas queries como clases de la ontología schema con esa propiedad existan.
\\\\
Haciendo un recorrido sobre los datos desde su origen (los documentos html publicados) hasta este punto, se encontró que el problema se 
originó en el proceso de extracción. 
Any23 cuando extrae las propiedades de la ontología schema, modifica las mismas adicionando el nombre de la clase de su correspondiente dominio, provocando que éstas queden incorrectas.

\section{Evaluación nº 2 - Duplicados}
\label{section:evaluacion-duplicados}

\subsection*{Objetivo}
Disponiendo de un gran dataset de información recolectada de la web, un paso muy útil es limpiar aquellos datos innecesarios. 
Este paso consiste en detectar aquellos reviews en el dataset que se encuentren duplicados.

\subsection*{Estrategia}
La forma más sencilla que a cualquiera normalmente se le ocurriría es comparar los reviews todos con todos. Cuyo algoritmo tiene orden cuadrático que para el tamaño del dataset resulta demasiado denso.
El algoritmo que se utilizó fue el siguiente:
\begin{enumerate}
\item Por cada ontología de review se escogió arbitrariamente las propiedades más utilizadas y representativas de cada una:

\textbf{Purl-Review}
\begin{itemize}
\item dcterms:date
\item rev:text
\item rev:title
\item rev:rating
\end{itemize}


\textbf{Schema-Review}
\begin{itemize}
\item sorg:datePublished
\item sorg:description
\item sorg:name
\item sorg:author
\item sorg:reviewBody
\end{itemize}

\textbf{Purl-ReviewAggregate}

\begin{itemize}
\item revagg:average
\item revagg:count
\item dcterms:date
\item rev:title
\end{itemize}

\textbf{Schema-AggregateRating}

\begin{itemize}
\item sorgagg:ratingCount
\item sorgagg:ratingValue
\item sorgagg:reviewCount
\end{itemize}

\item Luego para cada ontología se arma un mapa para cada propiedad escogida, que contiene como clave todos los objetos existentes para esa propiedad, y 
como valor un array con todos los reviews que contiene ese objeto para dicha propiedad.
En otras palabras se separaron por cada propiedad, los reviews según su valor para esa propiedad.
Luego se buscaron qué grupos de reviews se encontraban juntos para más de una propiedad mendiante obtener la intersección de ambos conjuntos. Por ejemplo:

Sea los reviews \{review1, review2, review3\} el conjunto perteneciente al valor 4 de la propiedad rev:rating 
Y sean los reviews \{review2, review3, review4\} el conjunto perteneciente al valor ``buena película'' de la propiedad rev:text 
La intersección entre ambos conjuntos es \{review2, review3\}. Si además todos los reviews cumplen que el ítem al cual evalúan 
tiene el mismo nombre en cada uno de los elementos, entonces serán marcados como potenciales duplicados.

Esta comparación se realiza entre todos los conjuntos de reviews de los valores de una propiedad con el resto de los conjuntos de los valores de las restantes propiedades de esa ontología.

\item Una vez obtenidos los conjuntos de los posibles duplicados se procede ahora sí a analizar minuciosamente los reviews todos con todos pero dentro del conjunto marcado como posible grupo de duplicados
para corroborar.
\end{enumerate}

\subsection*{Resultados}

Se encontraron en total 17343 reviews replicados en un total de 78705 reviews de los cuales.

\noindent 1009 estaban duplicados dentro de un mismo documento.

\noindent 64991 estaban duplicados entre distintos documento pero dentro de un mismo dominio

\noindent 12709 estaban duplicados en documentos pertenecientes a distintos dominios
\\\\
En el Cuadro \ref{table:DominioDup} se observa una lista con los dominios a los cuales se le encontraron más cantidad de reviews duplicados:
\begin{table}[h]
\begin{tabular}{| l | c | }\hline
Dominio & Cantidad de reviews duplicados \\\hline
4outof10.com &	21582\\
www.superpages.com &	18756\\
www.kollermedia.at &	9864\\
www.realtruck.com &	8372\\
ormigo.com &	8133\\
www.reptilecentre.com &	3440\\
www.querfood.de &	3275\\
www.carsurvey.org &	2766\\
www.chip.de &	1526\\
www.thewinecellarinsider.com &	1491\\\hline
\end{tabular}
\caption{Cantidad de documentos duplicados por dominio}
\label{table:DominioDup}
\end{table}

\section{Evaluación nº 3 - RDFUnit-Automatizado}
\label{section:evaluacion-automatizado}

\subsection*{Objetivo}
Encontrar todos los problemas sintácticos de los datos relevantes del dataset e identificar cuales pueden representar un problema 
para construir la aplicación.

\subsection*{Estrategia}
El framework crea test automáticamente a partir de las ontologías, buscando una amplia variedad de problemas posibles. Por lo que se 
instaló el framework, se lo proveyó de las ontologías relevantes y se corrieron los test sobre el dataset.

\subsection*{Resultados}

El cuadro \ref{table:RDFUnitSchemaAutomatic} muestra los principales errores encontrados para la ontología schema.
\begin{table}[h]
\begin{tabular}{| l | c | r | }\hline
Test & Fallos & Casos \\\hline
 schema:datePublished does not have datatype:& 152138 & 152138\\ XMLS:date & & \\\hline
 schema:ratingCount does not have datatype: & 12509 & 12509\\ XMLS:decimal & & \\\hline
 schema:ratingValue has rdfs:domain different from:& 6198 & 181178\\ schema:Rating & & \\\hline
 schema:worstRating has rdfs:domain different from:& 5867 & 108095\\ schema:Rating & & \\\hline
 schema:bestRating has rdfs:domain different from:& 5867 & 122057\\ schema:Rating & & \\\hline
 schema:aggregateRating is missing proper range & 4984 & 56890\\\hline
 schema:reviewRating has different range from: & 4095 & 144239\\ schema:Rating & & \\\hline
 schema:name does not contain a literal value & 1310 & 190840\\ (XMLS:string) & & \\\hline
 schema:itemReviewed is missing proper range & 1058 & 1505\\\hline
 schema:reviewRating has rdfs:domain different from:& 331 & 144239\\ schema:Review & & \\\hline
 schema:email does not contain a literal value & 174  & 174 \\ (XMLS:string) & & \\\hline
\end{tabular}
\caption{Resultados del test RDFUnit automático para la ontología Schema}
\label{table:RDFUnitSchemaAutomatic}
\end{table}

El Cuadro \ref{table:RDFUnitPurlAutomatic} muestra el resultado de correr el framework RDFUnit automático sobre la ontología de PURL.
\begin{table}[h]
\begin{tabular}{| l | c | r | }\hline
Test & Fallos & Casos \\\hline
 rev:reviewer is missing proper range & 83340 & 157004\\
 rev:reviewer has different range from: foaf:Person & 73664 & 157004\\
 rev:hasReview has different range from: rev:Review & 72398 & 347548\\
 rev:title has rdfs:domain different from: rev:Review & 14705 & 104900 \\ \hline
\end{tabular}
\caption{Resultados del test RDFUnit automático}
\label{table:RDFUnitPurlAutomatic}
\end{table}

\section{Evaluación nº 4 - RDFUnit-Manual}
\label{section:evaluacion-manual}

\subsection*{Objetivo} Correr test manuales en RDFUnit para evaluar el dataset en busca de errores sintácticos no contemplados por los test
automáticos de la evaluación anterior.

\subsection*{Estrategia} Se configuró RDFUnit utilizando patrones del framework para crear tests manuales. Luego se corrió para obtener resultados
de la misma manera que en la evaluación anterior

\subsection*{Resultados}

En el Cuadro \ref{table:RDFUnitManual} se visualizan los resultados de aplicar la evaluación manual.
\begin{table}[h]
\begin{tabular}{| l | c | r | }\hline
Test & Fallos & Casos \\\hline
http://purl.org/dc/terms/date has a & 92580 & 207130\\ format different to YYYY-MM-DD & & \\\hline
http://schema.org/datePublished has a & 53740 & 144594\\ format different to YYYY-MM-DD   & & \\\hline
http://schema.org/publishDate has a & 50070 & 59825\\ format different to YYYY-MM-DD  & & \\\hline
http://schema.org/dtreviewed has a & 26903 & 26903\\ format different to YYYY-MM-DD & & \\\hline
http://purl.org/stuff/rev\#rating is & 15319 & 298458\\ not in the expected range (1-5) & & \\\hline
http://purl.org/stuff/rev\#rating is & 2866 & 301322\\ not a natural number & & \\\hline
http://schema.org/ratingValue is & 41 & 108088 \\ smaller than http://schema.org/worstRating & & \\\hline
http://schema.org/ratingValue is & 1 & 73090 \\ not in the expected range (1-5) & & \\\hline
\end{tabular}
\caption{Resultados del test RDFUnit manual}
\label{table:RDFUnitManual}
\end{table}

\section{Evaluación nº 5 - Análisis experto}
\label{section:evaluacion-analisis}

\subsection*{Objetivo} Encontrar problemas a grandes rasgos que son muy difícil de detectar por algoritmos. También indicios indicadores de 
situaciones que lleven a pensar en otras búsquedas de problemas. Esto debería ayudar a entender mejor el dataset para mejorar el manejo 
de los datos en posteriores paso, como el de integración e implementación.

\subsection*{Estrategia} Este último paso de evaluación se comienza con un pantallazo general de cómo está la situación de los reviews. Para esto se pueden analizar las propiedades que contienen los reviews, y luego el valor objeto 
de las mismas. Dado que los problemas sintácticos deberían estar cubiertos en las evaluaciones anteriores, este paso sólo dejaría establecidos
algunos problemas semánticos del dataset, que al ser detectados en base al punto de vista racional humano, resultaría improbable poder encontrar
una solución algorítmica automatizada para el problema.

Primera query:

\begin{lstlisting}[frame=single]
SELECT ?p (count (distinct ?review) as ?cantidad) 
WHERE{
?review a {ReviewClass}.
?review ?p ?o .
}GROUP BY ?p ORDER BY DESC(?cantidad)
\end{lstlisting}

Esta query debería devolver las propiedades que poseen los reviews y la cantidad de reviews que la utilizan para una clase de review dada.

\subsection*{Resultados}

La clase Review de la ontología schema utilizaba 15 propiedades. 

\begin{table}[h]
\begin{tabular}{| l | c | }\hline
 Propiedad & Cantidad \\\hline
 http://schema.org/author & 146243 \\
 http://schema.org/datePublished & 144594 \\
 http://schema.org/reviewRating & 143908 \\
 http://schema.org/description & 122519 \\
 http://schema.org/name & 106219 \\
 http://schema.org/reviewBody & 93634 \\
 http://schema.org/publishDate & 59825 \\
 http://schema.org/url & 44905 \\
 http://schema.org/dtreviewed & 26903 \\
 http://schema.org/reviewer & 26903 \\
 http://schema.org/about & 14693 \\
 http://schema.org/keywords & 7119 \\
 http://schema.org/review & 2292 \\
 http://schema.org/itemReviewed & 2071 \\
 http://schema.org/itemreviewed & 67 \\\hline
\end{tabular}
\caption{Estadística de las propiedades de la clase schema:Review}
\label{table:PropertiesStatisticsSchema}
\end{table}

 \noindent Mirando el Cuadro \ref{table:PropertiesStatisticsSchema}, el primer error semántico que salta a la vista son las 2292 propiedades schema:review dentro un individuo de clase schema:Review. 
 Es claramente un error en la producción del documento. 
 
 Una rápida búsqueda de los documentos con este problema mostró que son dos los dominios que repiten el error a lo largo de sus documentos
 \\\noindent\textit{http://www.ilgiardinodeilibri.it/} y \textit{http://www.iplaysoft.com/}.
 Más en detalle la intención de los autores del documento fue realizar comentarios para los reviews (situación que tiene una solución específica 
 y es el uso de la propiedad schema:comments). 
 
 La causa principal de que existan errores de este tipo es la ontología schema que aunque no tenga sentido, propone que sea sintácticamente válido
 escribir un review de un review. Dado que la clase Review es subclase de Creative Work, y esta última contiene la propiedad review.
 \\\\
 Por otro lado otra propiedad en el listado que llama la atención es description. Analizando la propiedad semánticamente, significaría describir 
 el review en sí mismo, y no el ítem, dado que si ese fuese el objetivo, dicha propiedad estaría en el ítem y no en el review.
 Observando varios de los valores otorgados a esa propiedad en el contexto del review, muestran que se trata de la opinión textual del ítem, 
 para lo cual al igual que en la situación anterior existe una propiedad específica para eso: schema:reviewBody. 
 
 La causa de que se utilice schema:description en lugar de schema:reviewBody (incluso con mayor frecuencia) proviene de la ontología de data vocabulary, en la cuál 
 como se explicó al principio de la tesis utiliza la propiedad description como propiedad específica para realizar el review textual.
 Puede suceder que sean dominios que migraron sus reviews desde data vocabulary hacia schema y usaron como propiedad equivalente la que 
 era literalmente igual (lo cual es incorrecto). También puede deberse a la costumbre de haber utilizado aquella ontología que hizo que no se le 
 preste atención a este detalle, o simplemente que hayan visto que es también sintácticamente correcto realizar una descripción de un review 
 y creyeron que era correcto utilizarlo para plantear el review textual.
 De cualquier manera la ontología vuelve a generar inconvenientes debido a la jerarquía propuesta (Review hereda description de la clase Thing).
 
 Vale la pena aclarar que también existe otra propiedad proveniente de Creative Work que es schema:text cuya descripción de la ontología es `
 ``El contenido textual de este trabajo creativo'' la cuál sería sintáctica y semánticamente correcto utilizarla en lugar de reviewBody, aunque 
 no sería la más específica.
 \\\\
 La propiedad name, es ampliamente utilizada en los reviews, la descripción de la propiedad en la ontología declara que se refiere al nombre del objeto en cuestión. 
 
 Claramente no está pensada para un review (Al igual que la anterior se hereda de schema). Esta propiedad tiene sentido para algún objeto con un nombre real, como ``Teclado Genius L3'', o ``Batman Begins''.
 Un análisis en el uso de la propiedad bajo este contexto, demuestra que la intención fue utilizar esta propiedad para proveer un título al review.
 Y al no haber una propiedad provista por parte de la ontología para titular un review el uso de la propiedad name en su lugar si bien 
 es semánticamente incorrecto, no es una mala idea implementarlo.
 \\\\
 Puede observarse también que se utiliza la propiedad itemreviewed, la cual no existe en la ontología schema, pero claramente es un 
 error de tipeo en un intento de utilizar la propiedad itemReviewed. Éste al parecer es un error sintáctico no cubierto en las evaluaciones 
 anteriores.
 \\\\
 Una propiedad curiosa que aparece en el listado es schema:about. Esta propiedad debería describir el tema en el cual 
 se está tratando en el review. Pero un simple vistazo a los valores que toma la propiedad mediante una consulta SPARQL muestran que 
 el único valor es ``This is required''.
 \\\\
 Otra propiedad que vale la pena analizar es schema:keywords. Es una propiedad que debería contener palabras que describan al recurso, 
 el recurso en cuestión es el review y no el ítem, por lo que nuevamente nos encontramos con un error semántico en el uso de la propiedad (ya que es sintácticamente 
 correcto utilizarla en un review porque forma parte de la clase CreativeWork), pero no tiene sentido utilizar palabras para describir un review, claramente 
 son palabras que utilizaron para describir al ítem.
 
 Una observación de los valores verifican lo descripto anteriormente y además muestran que esta propiedad es sólo utilizada en los reviews por el dominio 
 \\\noindent http://www.kennstdueinen.de/ que además utiliza el idioma danés.
 \\\\
 También hay que mencionar la superpoblación de propiedades que establecen la fecha del review: datePublished, publishDate y dtreviewed.
 El motivo por el cual aparecen tres propiedades distintas para referirse a lo mismo es básicamente porque publishDate y dtreviewed no existen bajo la 
 ontología schema. Es el mismo caso que itemreviewed. 
 
 Es interesante también mencionar que tanto itemreviewed como dtreviewed son propiedades existentes bajo la ontología data vocabulary, 
 por lo que la causa es muy probable que sea a un intento erróneo de migrar el vocabulario de una a otra ontología, o de venir acostumbrado a la 
 ontología anterior y no prestar atención a este detalle.
 En cambio el motivo por el cual muchos dominios optaron por utilizar publishDate en lugar de datePublished, es que hasta fines de 2014, 
 era una propiedad válida que además se encontraba como ejemplo de un correcto uso de la clase Review provisto por parte de la documentación 
 de la ontología. 
 
 También la misma ontología en su documentación aclaraba que tanto datePublished como publishDate eran indistintas.
 Exactamente esto mismo sucede con la propiedad reviewer, la cual ya no está disponible.
 \\\\
 Por último, un vistazo a la propiedad más popular de la clase schema:Review, schema:author, mostró que todos los valores de dicha propiedad están compuestos 
 por un string o más específicamente un Literal. Esto contradice la definición de la ontología schema, tanto la documentación como la implementación, que
 establecen que le rango de la propiedad author es Organization o Person. 
 
 El asunto más relevante en este caso es entender porqué este error sintáctico de rango de propiedades no fue contemplado en los test 
 automáticos generados por el framework RDFUnit.
 Y la respuesta parece estar en la complejidad con la que establecieron el rango en la definición de la ontología. El extracto de las 
 tripletas relevantes muestran lo siguiente:
 \begin{lstlisting}[frame=single]
 schema:author rdfs:domain schema:CreativeWork .
 _:node1 rdf:type owl:Class .
 _:node2 rdf:first schema:Organization .
 _:node2 rdf:rest _:node3 .
 _:node3 rdf:first schema:Person .
 _:node3 rdf:rest rdf:nil .
 _:node1 owl:unionOf _:node2 .
 schema:author rdfs:range _:node1 .
 \end{lstlisting}
 
 Al parecer la definición del rango de la propiedad author utiliza estructuras de lista y conjuntos que parecen requerir más inferencia de la 
 que el framework RDFUnit puede realizar.
 \\\\
 Conclusión: La mayoría de los problemas son ocasionados por la inconsistencia, inestabilidad e incoherencia de la ontología. Principalmente 
 en el afán de hacerla elegante, crearon a Review como subclase de CreativeWork que normalmente suele ser superclase de clases que son ítems de reviews
 lo que concluye en una ontología demasiado larga y redundante.
 \\
 \\
 Para la clase purl:Review\\
 \begin{table}[h]
 \begin{tabular}{| l | c | }\hline
 Propiedad & Cantidad \\\hline
 \verb|http://purl.org/stuff/rev#text| & 207259\\
 \verb|http://purl.org/stuff/rev#rating| & 195005\\
 \verb|http://purl.org/stuff/rev#reviewer| & 154663\\
 \verb|http://purl.org/dc/terms/date| & 146473\\
 \verb|http://purl.org/stuff/rev#title| & 88148\\
 \verb|http://purl.org/stuff/rev#hasReview| & 6424\\
 \verb|http://www.w3.org/2006/vcard/ns#n| & 6424\\
 \verb|http://www.w3.org/2006/vcard/ns#fn| & 6424\\
 \verb|http://www.w3.org/2006/vcard/ns#url| & 6420\\
 \verb|http://www.w3.org/2006/vcard/ns#geo| & 6123\\\hline
\end{tabular}
\caption{Propiedades de la clase purl:Review}
\label{table:PropertiesStatisticsPurl}
\end{table}
\\ 

Los resultados más llamativos son la aparición de propiedades que normalmente se encuentran en el ítem, como puede observarse en el Cuadro \ref{table:PropertiesStatisticsPurl},
dado que no tienen sentido para un review que son hasReview (un review dentro de un review), n (nombre para un review), fn (idem n),
geo (posición geográfica de un review). El hecho de que todas se encuentren en cantidades similares es el gran indicador de que es 
un dominio repitiendo algún problema a lo largo de sus documentos. Una inspección de los dominios donde se encuentra cada uan muestra
que todas están incluídas en http://hoteles.hotelopia.es y http://hotels.hotelopia.it/ .
Básicamente el problema es que le asignaron el tipo purl:Review a ítems.
 
%Estrategia:

%Como base se utilizó el framework RDFUnit que provee funcionalidades que asisten tanto a la búsqueda de problemas sintácticos como semánticos.

%Para encontrar los errores sintácticos se utilizaró una herramienta del framework que genera test automáticos sobre un vocabulario, y esto se realizó sobre PURL y SCHEMA.

%RDFUnit también dispone de una serie de patrones de tests que pueden ser aplicados al contexto e implementados, algunos de los cuales pueden ser utilizados para detectar errores semánticos. 
