\chapter{Evaluación de Calidad de los Datos}
\label{chapter:evaluacion}

Objetivo:
Encontrar problemas en el dataset que dificulten una posible integración y modifiquen el correcto resultado de una posible aplicación derivada.

Estrategia:
La estrategia se divide en dos partes, una formal y otra informal.

La formal se encuentra ligada a la detección de errores sintácticos en el dataset, que son generalmente causados por una mala utilización del vocabulario (Aunque también suelen aparecer problemas causados por malas definiciones de los vocabularios)

El ejemplo típico de este tipo de error es el uso incorrecto del dominio o rango de una propiedad. 

Para realizar este procedimiento se utilizó el framework RDFUnit, que dispone de un conjunto de patrones de búsqueda de problemas y ademas permite definir los propios.

Y luego se encuentra la estrategia informal, que intenta encontrar los problemas semánticos del dataset, para los cuales resulta extremadamente difícil su detección automatizada.

Un ejemplo de este caso es la falta de información precisa necesaria para identificar un ítem, por ejemplo poner ``batman'' como nombre, siendo esta la única forma de identificarlo. Como sabemos batman puede referirse a muchos ítem distintos y se necesita un nombre más preciso para su identificación.
Este problema haría más dificultosa una posible integración, ya que no sería tan simple identificar cuales reviews hablan de los mismos ítems.

Otro ejemplo es el de la inconsistencia en la información provista, si por ejemplo se establece schema:Book como tipo de ítem pero el nombre del ítem es ``Samsung SyncMaster 753s'' claramente es el caso de un tipo incorrecto, ya que hay una propiedad que sólo tendría sentido si se tratara de un schema:Product.
Este último error modificaría el correcto funcionamiento de una posible aplicación resultante además de interferir en la integración. Ya que si esa aplicación por ejemplo, lista reviews según el tipo de ítem, cuando un usuario busque reviews de libros, se encontraría con el review de un monitor.

La manera en la que se implementó esta estrategia fue deducir posibles problemas en base a analizar resultados de consultas estadísticas de SPARQL sobre el dataset con el fin de ver humanamente qué problemas parecían generales.

También se buscó casos de problemas comunes, como por ejemplo los datos duplicados.