\chapter{Recolección de Datos}
\label{chapter:recoleccion}

Objetivo 

Como se indicó anteriormente la información semántica que va a ser necesaria para construir se encuentra en la web en forma de documentos 
HTML que tienen la particularidad de ser muy efímeros, de manera tal que un documento que poseía datos relevantes a la fecha,
puede al día siguiente, o dejar de estar disponible on-line, o haber cambiado de forma tal que la información de éste ya no es 
relevante, o ya no la posee. 

En [] sección 4.2 Challenges for the selection of data sources se generó una estadística de este caso, donde se estableció que 
en promedio 62\% de los documentos entoncontrados, continuaban on-line luego de un año, y de estos, sólo un 56\% aún poseían 
datos relevantes. 

Si bien armar un dataset con sólo información extraída de los documentos sin descargar estos últimos es posible, la situación anterior 
genera la necesidad de mantenerlos en una copia local para evitar una posible pérdida de los mismos. 

El objetivo entonces será armar un repositorio local con los documentos on-line descargados que se cree que tienen la información 
necesaria. 

 

Estrategia


Recolección y extracción de los datos
Como se mencionó antes, la web contiene grandes cantidades de documentos publicados con información semántica. Pero la tarea
de encontrarlos, con el agregado de que sólo una pequeña porción de ellos será relevante para los requerimientos no es trivial
en lo absoluto debido a la inmensidad del universo en el que se encuentran. La forma de llevar a cabo este objetivo está 
atada al hardware disponible, tanto para almacenar los datos, como para el tiempo que va a emplear la ejecución de esta 
tarea. 
Dado que las bases de datos semánticas sólo almacenan información en forma de tripletas o cuadrupletas, los documentos encontrados 
deberán someterse a un proceso de extracción que seleccione las sentencias HTML y las convierta a alguno de los lenguajes que soportan  
tripletas o cuadrupletas. Para esto existen múltiples herramientas.
Una vez transformados los documentos HTML a documentos semánticos puede construirse la base de datos semántica con la información 
recolectada.



