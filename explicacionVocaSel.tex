Tanto la recolección como la publicación de datos en la web semántica involucra la elección del o de los vocabulario/s que 
mejor modelen el dominio de problema, con la excepción que para su publicación existe la posibilidad de desarrollar uno propio 
que se ajuste correctamente en caso de que no se encuentre uno existente. \\
Seleccionar un vocabulario implica analizar varios aspectos del mismo, no sólo su definición e implementación, sino también 
el uso práctico dado por sus usuarios. \\
En primer lugar se debe comprobar que los nombres de las propiedades que posee sean correctamente autoexplicativas. Supongamos 
por ejemplo que existe un ítem con un rating agregado modelado por una ontología que posee las siguientes propiedades:\\
=minRating\\
=ratingValue\\
=countRating\\
Las dos últimas propiedades resultan facilmente identificables, ratingValue se trata del promedio de puntaje, y ratingCount 
la cantidad de puntajes que le fueron otorgados, pero la propiedad minRating podría generar distintas formas de interpretación, 
alguien podría suponer que se trata del valor mínimo que fue adquirido por un usuario, o el valor mínimo que un usuario puede 
otorgar. Y muchas veces la documentación de la ontología (si es que existe) no es suficiente.\\
Luego se deberá analizar si existen las propiedades para cubrir las necesidades mínimas de los casos de uso.\\
Y por último se debería intentar buscar ejemplos reales que muestren el uso que le dieron los usuarios a la ontología, para 
determinar qué propiedades están incorrectamente interpretadas o también para los casos donde las propiedades que se encuentren 
en desuso.\\
Con estas precauciones en mente se puede emprender la búsqueda, que podría tener como comienzo búsquedas en  search engines. 
Existen dos buscadores específicos para esta tarea.\\
\input{vocabcc}
\input{lov}