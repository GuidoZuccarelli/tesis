\part{Conclusiones}

\chapter{Conclusiones y trabajo futuro}
\label{chapter:conclusiones}

Se demostró cómo puede hacerse uso de las tecnologías, frameworks y estándares de la Web 
Semántica para explotar las contribuciones de los usuarios a lo largo de toda la web dentro 
del marco de los Sistemas de Recomendaciones.
\\\\
A partir de todo el proceso se pueden distinguir dos actores que participan en la publicación 
de reviews en la web semántica, el usuario que genera la información y el desarrollador
del sitio que proveé al usuario la plataforma para hacerlo.
\\\\
La importancia entonces de la existencia de vocabularios expresivos, concretos y precisos 
resulta fundamental para ayudar a estos dos a publicar la información correctamente, ya que 
se demostró que a partir del paso Evaluación, la incorrectitud en la información publicada 
crea grandes problemas que resultan muy complicados de resolver, al igual que se demostró 
que muchos de los errores son muy influenciados por los vocabularios.
\\\\
Aún con vocabularios completos, correctos y precisos, continuarían presentándose dificultades en 
el paso de Integración, ya que es muy dependiente de la relación entre los datos publicados 
por distintos usuarios.

Intentar integrar ítems y autores parece ser entonces una tarea que va de extremadamente compleja 
a imposible, dado que la mayoría de los reviews poseén de información de los autores sólo un username, 
en lugar de un identificador global como podría ser una dirección de email.

También ítems, que tampoco poseén identificadores globales, sino nombres escritos en 
distintos formatos.

Esto parece tener origen en la naturaleza decentralizada de la web semántica, que otorga 
a los usuarios la libertad de generar recursos sin límite.

La solución no parece encontrarse por el camino de la centralización, como fue planteado en \cite{Heath2006J} cuya
aplicación resultante terminó siendo obsoleta,
sino con la estandarización más restrictiva en vocabularios y propiedades, intentando 
llevar a los usuarios a la utilización de esos tan necesarios identificadores.
\\\\
Por otro lado se pudo observar la preferencia de los desarrolladores de sitios a la utilización de 
formatos de código semántico embebido en HTML, en lugar de documentos RDF puros. Aunque cada uno de 
los formatos poseé ventajas y desventajas para dicho desarrollador del sitio, se notó que uno de ellos 
(microformatos) sólo contnía desventajas desde el punto de vista del desarrollador de la aplicación 
semántica. 

Dicha conclusión  se obtiene de verificar que la calidad de los ítems escritos en microformatos del 
dataset es muy baja, ya que sólo poseén como información un nombre en la mayoría de sus casos.

Esto se debe a que no se disponen de namespaces para la utilización de ontologías que permitan 
la utilización de propiedades descriptivas necesarias, y sólo cuenten con el uso de V-Cards.

El punto de vista del desarrollador de aplciaciones semánticas por supuesto que no suele ser tenido 
en cuenta por el desarrollador de sitios, que la mayoría de las veces opta por el uso de 
microformatos por su simplicidad.
\\\\
A pesar de los problemas mencionados, se alcanzó el objetivo de construir un sistema de recomendación, 
dondé se produjeron satisfactorios resultados en los puntos de evaluación y curación. 
Y dicha construcción demostró cómo puede ser necesarios combinar vocabularios (como se vió en 
el capítulo \ref{chapter:integracion} que se aprovechó la simplicidad y correctitud de la 
Review Ontology para los reviews para los ítems de schema), y mostró interesantes usos 
de la minería de datos para el proceso.

También mostró la necesidad de conocer las distintas herramientas y aplicaciones como Sindice 
o RDFUnit, que aportan mucho al proceso, también de la necesidad de conocer la naturaleza de 
los datos con los que se trabaja.
\\\\
Quedan entonces planteados, desafíos en varios de los puntos de este trabajo, comenzando con 
la proposición de estándares o soluciones que resuelvan el problema de la publicación de información 
no identificable, siguiendo por el análisis del uso de la minería de datos en varios puntos, y 
la generación de una estrategia de integración exitosa en este contexto. 

Otro aspecto a investigar es el de la obtención de fuentes de datos, más completa y actual que
no necesite de la utilización de complejos recur5sos de hardware para un crawling, dado 
que Sindice a partir de 2015 ha dejado de funcionar como web crawler.

Un punto interesante de partida podría ser la inspección del nuevo LOD Laudromatic.

Por último también queda por mejorar los puntos de Evaluación y Curación, que pueden analizarse 
cuánto más completo y eficiente pueden lograrse.
