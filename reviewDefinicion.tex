El review es una expresión de la persona que utilizó el ítem que refleja su grado de conformidad con el mismo, con el objetivo de informarlo a otras personas. Por lo que entonces, el atributo mínimo será aquel que refleje este sentimiento del usuario hacia el ítem, el mismo puede ser un texto explicativo (por ejemplo \"Una pel'icula hermosa, pero me parecieron flojos lso actores\") o un valor num'erico dentro de un rango determinado (7 en escala de 1 a 10).\\
A partir del atributo base, existen muchos otros atributos que, bien construidos, aportan mucha riqueza al aprovechamiento del review. Por ejemplo, el autor, que será un atributo que identifica a la persona creadora del review. Otro caso es el de la fecha.\\