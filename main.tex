
\documentclass[10pt]{report}
\pagestyle{plain}
\usepackage[spanish]{babel}
\selectlanguage{spanish}
\usepackage[utf8]{inputenc}
\usepackage{listings}
\usepackage{color}
\usepackage[table]{xcolor}
\usepackage{graphicx}
\usepackage{amssymb}
\usepackage{appendix}
\usepackage[colorlinks=true, pdfstartview=FitV, linkcolor=blue, 
            citecolor=blue, urlcolor=blue]{hyperref}
\usepackage{framed}
\DeclareUnicodeCharacter{00BD}{~}

%opening
\title{La Web Semántica como plataforma para sistemas de recomendación}
\author{Guido Zuccarelli}


\begin{document}


\maketitle

\tableofcontents

%Cada uno de los que siguen a continuación es un chapter
\part{Introducción}

\chapter{Motivación}
\label{chapter:introduccion}

\section{Evaluaciones y opiniones}
\label{section:evaluaciones-cruzadas}

\subsection{Ítem}
\noindent Llamaremos ítem a un elemento del mundo real que es aprovechado por las personas, dicho ítem puede ser un objeto, un servicio, una idea, un programa, etc. Además, debe poder ser abstraído a un modelo representado por una computadora, de manera tal que esa representación describa con precisión a qué se refiere ante la interpretación del lector. 

\noindent Para ello, el modelo del ítem contendrá un conjunto de atributos que lo describen. Algunos más importantes que otros.
\\\\
Por ejemplo, imaginemos que se debe representar una bicicleta mediante un conjunto de pares atributo;valor, podría utilizarse una representación como la siguiente:
\begin{lstlisting}[frame=single] 
Tipo: Bicicleta
Sexo: Unisex
Talle: 51
Cuadro-Material: Aluminio
Cuadro-color: Rojo
Cuadro-Tipo: Ruta
Horquilla-Material: Fibra de carbono
Horquilla-Color: Rojo
Asiento-tipo: Adamo 
Asiento-Color: Blanco
\end{lstlisting}
La cantidad de atributos que pueden utilizarse para describir el ítem puede ampliarse ilimitadamente. 
\\\\\\
Reducir la cantidad de pares de atributos para describir el mismo ítem es posible de la siguiente manera: 
\begin{lstlisting}[frame=single] 
Tipo: Bicicleta
Talle: 51 
Marca: Merida
Modelo: Reacto 500
\end{lstlisting}
En este caso se aprovechó el hecho de que ya existan definiciones del ítem que intentamos describir, por lo que si se utilizan sólo los atributos que lo identifican, será suficiente para que el lector tenga la interpretación correcta del mismo. 
\\\\
\subsection{Usuario}
Las personas cotidianamente hacen uso de estos ítems, convirtiéndolos en usuarios del mismo. Y estos usuarios pueden de la misma manera ser computacionalmente representados mediante pares de atributo;valor. Y al igual que los ítems algunos atributos servirán para identificar al usuario. 
\\\\
\subsection{Reseña}
Definiremos a una reseña (de aquí en adelante review) a la representación mediante la computadora, de la medida de conformidad con respecto a dicha relación entre usuario e ítem. 

Esta relación de uso entre usuario e ítem que puede observarse en la figura \ref{figure:reviewItemUser}, contiene un grado de conformidad entre el primero y el segundo, si el ítem cumplió o no con las expectativas del usuario debería poder modelarse también para ser representado computacionalmente.

El objetivo de un review, es que un usuario pueda reflejar el sentimiento que le generó utilizar el ítem e informarlo a otros usuarios. 

También los reviews dispondrán de un conjunto de atributos de los cuales algunos serán indispensables y otros que enriquecerán no sólo su valor intrínseco, sino también su utilidad para el contexto en el que será planteado.
\\\\
Imaginemos un usuario que realiza un review sobre la bicicleta, podría generar el siguiente conjunto de pares atributo-valor:
\begin{lstlisting}[frame=single] 
Puntuacion:4 
Titulo:``Buena opcion''
Texto:``Liviana y comoda, pero un poco rigida para 
        doblar, y no frena adecuadamente ''
Fecha:15/10/2012
\end{lstlisting}
\begin{figure}
    \centering
    \includegraphics[width=0.8\textwidth,natwidth=610,natheight=642]{biciReview.png}
    \caption{Relación entre usuario, review e ítem}
    \label{figure:reviewItemUser}
\end{figure}

La existencia de los reviews en la web es muy importante, ya que dan un panorama socialmente perfeccionado (dado que permiten que no sólo una persona
opine sobre un ítem, lo que da a lugar a muchas evaluaciones y provoca que se acerque al valor real) sobre un ítem, característica que ayuda a 
quienes necesiten una descripción y valoración de los mismos que no provenga de quien los quiere promocionar.

\section{Sistemas de Recomendación}
\label{section:sistemas-de-recomendacion}
\noindent Los sistemas de recomendación son herramientas de software que, en base a un conjunto de ítems (películas, libros, productos, hoteles, etc) e información sobre estos, y un conjunto de usuarios, intentan sugerir ítems apropiados a dichos usuarios \cite{Systems2011}. Puede verse en la figura \ref{figure:flujo} cómo estos se componen. Los sistemas de recomendación se han vuelto una de las herramientas más poderosas para múltiples tipos de aplicaciones web, como comercio electrónico o páginas de noticias. 
\\\\
El desarrollo de estas herramientas, involucra conocimiento en múltiples áreas, como inteligencia artificial, minería de datos, estadística, etc. 
\\\\
Los sistemas de recomendación poseen tres enfoques, el de filtrado colaborativo, el basado en contenido y el no personalizado.


\begin{figure}
    \centering
    \includegraphics[width=0.8\textwidth,natwidth=610,natheight=642]{recSis}
    \caption{Flujo de un Sistema de Recomendación}
    \label{figure:flujo}
\end{figure}

\subsection{Sistemas de recomendación basados en contenido}
Los sistemas de recomendación basados en contenido utilizan un conjunto de informaciones y descripciones de los ítems previamente valorados por un usuario para poder construir un perfil del mismo de manera tal de poder determinar, cuales son sus intereses  \cite{Systems2011}.

Una vez construido el perfil, pueden procesarse las características de distintos ítems que potencialmente pueden ser recomendados a dicho usuario para determinar si alguno de ellos va acorde al perfil.
\subsection{Sistemas de recomendación de filtrado colaborativo}
Los sistemas de recomendación de filtrado colaborativo utilizan la información histórica de cada usuario para intentar encontrar y generar grupos de usuarios con gustos similares \cite{Systems2011}. Para lograrlo se compara cada usuario con otro observando qué ítems evaluaron y qué puntajes fueron otorgados. 

De esta forma, si se requiere predecir el interés de un usuario en un ítem, se podrá buscar en dicho grupo de usuarios con gustos similares e inspeccionar aquellos usuarios que hayan realizado un review sobre el ítem.
\\\\
Los sistemas de recomendación de filtrado colaborativo tienen una eficacia superior a los basados en contenido, pero cuentan con una gran desventaja, no permiten predecir intereses para aquellos ítems que aún no poseen evaluaciones, o también para aquellos usuarios que no han evaluado ningún ítem.  A esto se lo llama “arranque en frío”.
\subsection{Sistemas de recomendación no personalizados}

Éste es el más simple de los sistemas de recomendación, ya que como refiere su nombre, no 
toma en cuenta las características de los usuarios.\\
Su funcionalidad se basa en el background de información obtenida sobre los ítems, recomendándolos
indistintamente a cada uno de los usuarios habiendo previamente generado un ranking de popularidad \cite{Poriya2014}.
Opcionalmente puede generarse más de un ranking separando los tipos de ítems (Por ejemplo tener un ranking de 
películas y otro de libros).\\
\\
Estos sistemas de recomendación son muy populares entre los eCommerce.
\section{La web semántica}
\label{section:la-web-semantica}
En sus comienzos en los 90, la web podía verse como un conjunto de sitios web que ofrecían una colección de documentos web interconectados mediante la hipermedia, con el objetivo de comunicar información a los usuarios.
El contenido de esos documentos sólo era generado por el mismo creador y publicador del documento y los usuarios se limitaban a consumirlo. Por otro lado, era a su vez estático, es decir, se publicaba en la misma forma que se almacenaba y no cambiaba.  
\\\\
Hacia fines de los 90 los sitios web comenzaron a implementar una serie de herramientas (que si bien ya se encontraban disponibles anteriormente no se utilizaban por un problema de performance) que permitieron a los usuarios finalmente participar de la producción del contenido web. Lo que produjo notorios cambios en cuanto a la cantidad de información disponible y proveyó diferentes formas de uso de la web (blogs, redes sociales, canales rss, etc). Más adelante ante la apreciación del pasaje de web estática a una web dinámica, se acuño a esa actual web como web 2.0 y retrónimamente 1.0 a la anterior.
\\\\
Ese cambio provocó un aumento en el tamaño de la web, que se volvió inmensamente grande, y llevó a la necesidad de implementar tecnologías que ayuden al aprovechamiento de esa cantidad de información. 
Se comenzó entonces a utilizar una serie de frameworks y estándares que permitieron enriquecer mediante metadatos semánticos y ontológicos dentro de los estándares de la W3C los datos contenidos en los documentos de manera tal que estos puedan ser consumidos, interpretados y utilizados directamente no sólo por las personas, sino también por las computadoras.
Esto generó que los datos también puedan ser relacionados entre sí, de la misma manera que los documentos son interconectados formando una web de documentos, los datos interconectados forman una web de datos paralela\cite{DiNoia2012}.
Todo este conjunto de actividades frameworks y herramientas forman la ``Web semántica'' que es el puntapié inicial para una web mucho más interoperable, lo que permite facilidades para el uso de la web por parte de las aplicaciones y da lugar a otro paso en la evolución de la web, la web 3.0. 
La figura \ref{figure:webevolution} muestra ejemplos de las tecnologías más importantes que marcaron cada uno de los puntos de evolución recientemente mencionados.

La Web Semántica promete facilitar el desarrollo de la web social y la inteligencia colectiva, a través de mecanismos de clasificación, relación y descripción del contenido de la información publicada mejorando su interoperabilidad \cite{Antoniou}. Promete también mejorar la recolección, agregación e integración de datos específicos mediante el uso de agentes de búsqueda automáticos que aprovechan las ventajas.
%Con el correr de los años, múltiples tecnologías se fueron implementando y permitieron el desarrollo de una web mucho más grande y aprovechable...  y esto es una referencia a un libro sobre web semántica \cite{Antoniou}

\begin{figure}
    \centering
    \includegraphics[width=0.8\textwidth,natwidth=610,natheight=642]{webevolve}
    \caption{Evolución de la web}
    \label{figure:webevolution}
\end{figure}

\section{Reviews en la web semántica}
\label{section:reviews-en-la-web}

La web 2.0 dio la posibilidad a los usuarios consumidores de la web de generar y publicar contenido en la misma, lo cual cumple con los requerimientos de una plataforma para reviews. Proporciona un entorno para crearlos y publicarlos, siendo está última una tarea muy sencilla. 

Pero la dificultad de encontrarlos y explotarlos parece ser inversamente proporcional a a la de generarlos. Dado que a mayor facilidad de publicar contenido en la web, mayor se vuelve la inmensidad de la misma y mayor se vuelve la dificultad de encontrar algo dentro de ella.

Veamos dos ejemplos que requieren encontrar reviews en la web y cómo la web semántica puede facilitarlo:

Imagínense que son ingenieros de Mérida y lanzaron al mercado la bicicleta Reacto 500. Como buenos ingenieros necesitan conocer qué opinan sus clientes, por lo que buscarán reviews en algún motor de búsqueda y luego leerán uno por uno cada review con el objetivo de resumir las opiniones. Humanamente realizar esta tarea para unos pocos reviews podría demandar mucho tiempo. 
Incluso peor aún, imagínense que necesitan saber que opina la gente de determinada región (por ejemplo el norte de Europa), la tarea de buscar los reviews necesarios se volvería aún más complicada.
Con la posibilidad de contar con una web en la cual los datos son interpretados por las aplicaciones de software y estos a su vez están interconectados, podría crearse una aplicación que pueda automáticamente buscar, clasificar y procesar los reviews para generar automáticamente el resumen.

Ahora bien, si en lugar de requerir reviews de un ítem en particular, se necesitan reviews para una aplicación que recomiende ítems a usuarios, ya no sería una tarea realizable humanamente. Para ello haría falta una aplicación que haga crawling en la web y de alguna manera identifique reviews y a su vez identifique el ítem al cual el review hace referencia. 
Parece algo muy difícil de lograr, aún trabajando sobre sitios conocidos con documentos estructuralmente dominados los cuales se pueda recorrer el DOM automáticamente y acceder a los reviews.
De nuevo, la Web Semántica promete solucionar este problema \cite{Heitmann}, con el uso de metadatos que dan información sobre los datos, haciendo que una aplicación pueda fácilmente identificar reviews y navegar por la hipermedia de los datos para conseguir información sobre el ítem y usuario referenciados \cite{Zhou2005}.

La figura \ref{figure:semanticwebreview} muestra un ejemplo concreto de un review publicado en la web, respetando los lineamientos de la web semántica. 

\begin{figure}
    \centering
    \includegraphics[width=0.8\textwidth,natwidth=610,natheight=642]{semanticwebreview}
    \caption{Ejemplo de review en la web semántica}
    \label{figure:semanticwebreview}
\end{figure}

En \cite{Heath2006} se presenta un sistema con el objetivo de publicar reviews en la web semántica. Dicho sistema no tuvo éxito
y la funcionalidad del mismo se encuentra actualmente inoperable.

\section{Caso de estudio}
\label{section:caso-de-estudio}
Para mantener en contexto este trabajo, utilizaremos un caso de estudio. Se trata de una aplicación que integra información de reviews disponible en la web semántica para producir recomendaciones simples. En particular, ofrece los siguientes servicios: 

\begin{description}
\item[Rankings de calidad ] Los rankings de calidad son listas de ítems ordenadas por el promedio de puntos 
obtenidos de sus reviews. Para construirlos en necesario obtener y combinar los puntajes de cada uno
de los reviews de cada ítem.
\item[Rankings de popularidad] Los rankings de popularidad son listas de ítems ordenadas por la cantidad
de reviews que posee cada ítem.
\item[Listas de tipos] Las listas de tipos son listas de los distintos tipos de ítem encontrados. 
Por ejemplo, películas, libros, productos, etc.
\item[Listas de ítems] Las listas de ítems son listas resultantes de búsquedas de ítems por nombre o por tipo. 
Para su construcción es necesario contar con los nombres o tipo de cada ítem.
\item[Filtro] Los filtros son opciones de filtrado de los rankings, para reducir la cantidad 
de ítems retornados por ellos. De manera que se puedan seleccionar ítems por fecha de review.
\item[Descripciones] Las descripciones son los detalles de los ítems, que incluyen toda la información que se dispone de estos 
incluyendo sus reviews. El usuario podrá recibir esta información proveyendo al sistema el id del ítem el cual se quiere consultar.
 
\end{description}


\section{Organización}
\label{section:organizacion}

 El capítulo \ref{chapter:semanticwebelements} introduce los conceptos de Web Semantica, sus principios y tecnologías y para que el capítulo \ref{chapter:estrategia} presente la estrategia de trabajo en términos generales. Los capítulos desde \ref{chapter:seleccion} a \ref{chapter:explotacion} discuten en detalle cada uno de pasos de la estrategia elegida, aplicándolos al caso de estudio. Finalmente, el capítulo \ref{chapter:conclusiones} resumen los resultados observados, saca conclusiones al respecto, y plantea trabajo futuro. El anexo \ref{chapter:publications} presenta las publicaciones que fueron resultado del trabajo efectuado en este trabajo de tesis. 





\chapter{Enfoque General}
\label{chapter:estrategia}

Lo que vamos a construir es un caso de aplicación de la web semantica....

\section{Aplicaciones en la Web Semantica}
\section{Principios}
Dar historia (muy breve), definiciones  (RDF, tripletas, owl, inferencia, ..)
Escribir a medida que lo necesitas-.

\section{Tecnologías}
Triplestores, frameworks, extractores, crawlers, motores de busqueda semanticos...

\section{Estrategia propuesta}
Con el fin de crear una aplicación que satisfaga los requerimientos mensionados anteriormente, se debe encontrar
un procedimiento que incluya desde obtener los datos relevantes de la web hasta llevarlos a un estado que permita una 
explotación satisfactoria. 

El procedimiento debería incluir los siguientes pasos  

\begin{description}
\item[Selección de vocabularios ] Tanto la recolección como la publicación de datos en la web semántica involucra la elección del o de los vocabulario/s que mejor modelen el dominio de problema, con la excepción que para su publicación existe la posibilidad de desarrollar uno propio que se ajuste correctamente en caso de que no se encuentre uno existente. 
\item[Recolección y extracción de los datos ] .... ?
\item[Evaluación de Calidad de los Datos] ???
\item[Curado de los Datos] ???
\item[Integración de los datos] ?? 
\item[Publicación del Dataset Curado] ?? 
\item[Explotación del Dataset] ?? 
\end{description}

\begin{framed}
\textcolor{red}{acá sería bueno incluir un diagrama que muestre el pipeline. En la lista de arriba con un párrafo se explica cada paso. Luego, en las secciones que sigue se analiza mas cada paso, pero se lo plantea como un problema.. se analizan los retos; entonces los capítulos 4 a 9, explican como los resolviste. Alternativamente, se puede hablar menos acá, solo lo suficiente para que se entienda la estrategia general, y luego en los capítulos 4 a 9 se analiza el problema en detalle y se propone la solución. Con esta ultima forma, todo lo que se refiere a los vocabularios quedaría en el capitulo 4 y no acá. Puede ser mejor.}
\end{framed}


\section{Selección de vocabularios}

Seleccionar un vocabulario implica analizar varios aspectos del mismo, no sólo su definición e implementación, sino también el uso práctico dado por sus usuarios. 
En primer lugar se debe comprobar que los nombres de las propiedades que posee sean correctamente auto-explicativas. Supongamos por ejemplo que existe un ítem con un rating agregado modelado por una ontología que posee las siguientes propiedades:

\begin{enumerate}
\item minrating
\item ratingValue
\item countRating
\end{enumerate}


Las dos últimas propiedades resultan fácilmente identificables, ratingValue se trata del promedio de puntaje, y ratingCount 
la cantidad de puntajes que le fueron otorgados, pero la propiedad minRating podría generar distintas formas de interpretación, 
alguien podría suponer que se trata del valor mínimo que fue adquirido por un usuario, o el valor mínimo que un usuario puede 
otorgar. Y muchas veces la documentación de la ontología (si es que existe) no es suficiente.

Luego se deberá analizar si existen las propiedades para cubrir las necesidades mínimas de los casos de uso.

Y por último se debería intentar buscar ejemplos reales que muestren el uso que le dieron los usuarios a la ontología, para 
determinar qué propiedades están incorrectamente interpretadas o también para los casos donde las propiedades que se encuentren 
en desuso.

Con estas precauciones en mente se puede emprender la búsqueda, que podría tener como comienzo búsquedas en  search engines. 
Existen dos buscadores específicos para esta tarea:

\subsection{lov}

Linked Open Vocabulary (LOV)

Proporciona una plataforma técnica de búsqueda y evaluación de calidad sobre un dataset extraído de linked data cloud que contiene descripciones de vocabularios RDFS 
y ontologías OWL. Esas descripciones están en forma de metadatos y pueden ser generados tanto por los autores de los vocabularios como por curadores de LOV.
Posee además de la búsqueda las funciones de estadística o sugerencia.

Actualmente el dataset está integrado por 475 namespaces distintos que contienen una media de 10 clases y 20 propiedades, siendo schema.org el más grande de ellos.

\subsection{vocabcc}

Vocab.cc

Vocabcc es un proyecto opensource que permite a los desarolladores de RDF realizar búsquedas de vocabularios de Linked Data.

Para facilitar la decisión de seleccionar un vocabulario deseado, proporciona además información estadística de cada uno sobre el dataset Billon Triple Challenge (BTCD). 
 Esta información incluye el número de apariciones globales de la URI dada en el BTCD, así como el número de documentos dentro de la BTCD, que contiene el URI dado. Estos números permiten una clasificación de propiedades y clases, respectivamente, con respecto a su uso. También se proporciona información acerca de la posición de una URI dada en estos rankings.

Los desarrolladores pueden buscar las URIs con queries arbitrarias o búsquedas de URIs específicos (prefijos comunes se resuelven automáticamente con datos de prefix.cc).

Para permitir una fácil integración de la funcionalidad vocab.cc, toda la información está disponible como RDF y se puede acceder como Servicio Vinculado.

\section{Recolección y extracción de los datos}

%Como se mencionó antes, la web contiene grandes cantidades de documentos publicados con información semántica. Pero la tarea
de encontrarlos, con el agregado de que sólo una pequeña porción de ellos será relevante para los requerimientos no es trivial
en lo absoluto debido a la inmensidad del universo en el que se encuentran. La forma de llevar a cabo este objetivo está 
atada al hardware disponible, tanto para almacenar los datos, como para el tiempo que va a emplear la ejecución de esta 
tarea. \\
Dado que las bases de datos semánticas sólo almacenan información en forma de tripletas o cuadrupletas, los documentos encontrados 
deberán someterse a un proceso de extracción que seleccione las sentencias HTML y las convierta a alguno de los lenguajes que soportan  
tripletas o cuadrupletas. Para esto existen múltiples herramientas.\\
Una vez encontrados, descargados y transformados los documentos HTML a documentos semánticos puede construirse la base de datos semántica 
con la información recolectada.\\
Estos son los cuatro pasos necesarios para lograr tener el dataset semántico con el cual se puede empezar a trabajar. Cada uno de ellos 
posee distintas alternativas para su realización, algunas se describirán a continuación\\
Como se mencionó antes, la web contiene grandes cantidades de documentos publicados con información semántica. Pero la tarea
de encontrarlos, con el agregado de que sólo una pequeña porción de ellos será relevante para los requerimientos no es trivial
en lo absoluto debido a la inmensidad del universo en el que se encuentran. La forma de llevar a cabo este objetivo está 
atada al hardware disponible, tanto para almacenar los datos, como para el tiempo que va a emplear la ejecución de esta 
tarea.  

Dado que las bases de datos semánticas sólo almacenan información en forma de tripletas o cuadrupletas, los documentos encontrados 
deberán someterse a un proceso de extracción que seleccione las sentencias HTML y las convierta a alguno de los lenguajes que soportan  
tripletas o cuadrupletas. Para esto existen múltiples herramientas. 

Una vez encontrados, descargados y transformados los documentos HTML a documentos semánticos puede construirse la base de datos semántica 
con la información recolectada. 

Estos son los cuatro pasos necesarios para lograr tener el dataset semántico con el cual se puede empezar a trabajar. Cada uno de ellos 
posee distintas alternativas para su realización, algunas se describirán a continuación 


%\input{opcionesBusqueda}
\\
\input{opcionesDescarga}
\\
\input{opcionesTransformación}
\\
\input{opcionesAlmacenaje} vacio

\subsection{Búsqueda}
%\input{opcionesBusqueda}
\textcolor{red}{OJO: ESTO ESTABA EN EL ARCHIVO busqueda.tex no en el opcionesBusqueda}


La forma de ejecución de esta tarea dependerá de algunos aspectos:

Recursos de hardware disponibles

Cantidad y calidad de la información requerida

El grado de atemporalidad mínima tolerable en los datos

El primer paso para realizar la recolección es intentar responder la siguiente pregunta: ¿Dónde encuentro la informanción?

Una vez seleccionados los vocabularios, se necesitará obtener fuentes de datos que contengan sus datos publicados en esos vocabularios.
 
Para lograrlo se podrá utilizar como punto de partida:

Sitios indexadores: Son algunos sitios que disponen de un dataset muy grande procesado con documentos indexados, que ofrecen consultar dicho 
dataset mediante servicios web. Generalmente proveen una API donde se pueden consutlar los datos mediante distintos grados de flexibilidad.

Sindice, LOD cloud cache y UriBurner son algunos ejemplos de estos sitios. Se puede utilizar entocnes, los servicios web a fin
de obtener una lista de URL que se cree que tendrán la información necesaria.

Sindice: Es una herramienta creada en conjunto por la Universidad de Deri,  Fondazione Bruno Kessler y Openlink Software que propociona múltiples tipos de API ofreciendo acceso a su dataset de la web de datos. Este dataset contiene información recolectada de la web en múltiples formatos de la web semántica y puede ser accedido a travez de un search engine, una API restful o un SPARQL endpoint.  
En la actualidad posee indexados 708.26 millones de documentos.

Link Open Data Cloud Cache (LOD): Al igual que sindice proporciona acceso a un dataset recolectado de la web de datos pero de una manera mucho más acotada. Se dispone de un text search (funciona como un search engine) o de un SPARQL Endpoint bastante restringido. Sólo se puede acceder al dataset entero mediante federated queries, en caso de querer buscar sobre todo el dataset, el endpoint limita la consulta sólo a una parte del mismo. Tambien posee muchas restricciones respecto a los time outs, por lo que las consultas no pueden ser muy flexibles.
Para poder disponer de la funcionalidad completa del endpoint y así poder aprovechar tanto el dataset como las consultas SPARQL deberá comrprarse una licencia.

Sitios autoritativos: Son sitios conocidos que generan información relevante y la publican en las ontologías y vocabularios 
seleccionados. Ejemplos aplicados al caso de estudio podrían ser IMDB (que publica reviews de películas bajo el vocabulario schema), o Rottentomatoes
(que hace lo mismo pero no solo con películas). Se podrán utilizar entonces estos sitios como punto base para un posible crawling.

Catálogos de enpoints: Son catálogos provistos por algunos sitios que mantienen una lista actualizada de SPARQL endpoints 
junto con el estado de disponibilidad en el que se encuentran. Por ejemplo los sitios http://labs.mondeca.com/sparqlendpointsstatus/ y 
http://www.w3.org/wiki/SparqlEndpoints que este último además provee detalles sobre la información de los mismos.

Consultar estos sitios entonces generará una lista de endpoints SPARQL que se podrá utilizar para consultar la información necesaria.

Volcados de datos: Son datasets muy grandes que fueron el resultado de un web crawling, los cuales están disponibles para su descarga 
y se pueden utilizar para procesarlos y obtener los datos requeridos. El más importante es http://challenge.semanticweb.org/2014/ .



%\input{opcionesDescarga}
%input{opcionesTransformación}
%input{opcionesAlmacenaje}

 

\section{Evaluación de Calidad de los Datos} 
Como sabemos los reviews son creados por usuarios que en su mayoría no están familiarizados con el desarrollo de una 
aplicación, esto causa que una parte muy significativa del contenido publicado no esté de la forma adecuada para ser procesado. 
La falta de calidad en el contenido publicado puede deberse tanto a errores por parte del usuario como por parte del publicador.


El publicador deberá asegurarse que los datos respeten rigurosamente las ontologías en las cuales se publican.

El sitio web en el cual el usuario se encuentre realizando el review deberá guiarlo en todo lo posible para lograr que la evaluación
quede en un formato adecuado.


Aunque también existen muchos problemas que no dependen del sitio web, generalmente errores semánticos de calidad, donde lo 
redactado esta hecho de forma inconsistente o insuficiente.


Este paso entonces tiene por objetivo encontrar todos los problemas de calidad que pueda haber en el dataset que acaba de 
ser descargado y extraído y que generen inconvenientes para una posterior integración/explotación.

%Curado de los Datos\\
En el paso anterior se describieron los problemas en la calidad del dataset que tendrán un impacto en la aplicación resultante, 
o que, podrían limitar o imposibilitar la realización de la misma. Dichos problemas pueden ser muchos, y algunos muy difíciles
de solucionar, será parte entonces del proceso, reconocer aquellos que su resolución sea viable y además encontrar la forma de 
implementarla.\\
Cabe destacar que el proceso de evaluar el dataset en búsqueda de problemas y implementar soluciones es iterativo, debido a que 
una mejora puede conllevar a nuevos problemas.\\
\section{Curado de los Datos}

En el paso anterior se describieron los problemas en la calidad del dataset que tendrán un impacto en la aplicación resultante, 
o que, podrían limitar o imposibilitar la realización de la misma. Dichos problemas pueden ser muchos, y algunos muy difíciles
de solucionar, será parte entonces del proceso, reconocer aquellos que su resolución sea viable y además encontrar la forma de 
implementarla.

Cabe destacar que el proceso de evaluar el dataset en búsqueda de problemas y implementar soluciones es iterativo, debido a que 
una mejora puede conllevar a nuevos problemas.

%Integración de los datos\\
Los datos recolectados fueron generados por distintos usuarios, en múltiples sitios y bajo distintas ontologías y estándares.
Estos aún curados, necesitan un último paso para poder realizar una explotación, y es la integración.\\
Este gran y dificultoso proceso abarca cualquier operación que intente dar una visión más unificada de la información. Este
proceso puede tener distintos niveles y aspectos a integrar, como podría ser por ejemplo, unificar las ontologías de los reviews, o bien
en una nueva ontología de review, o bien en una ya existente, para lograr tener información semánticamente más parecida. Ya 
que este proceso puede ser muy dificultoso, habrá que ver en base a los requerimientos, qué aspectos de los datos se integrarán.\\
Una vez realizada la integración mínima necesaria par also requerimientos, el dataset ya estará listo para su explotación.

\section{Integración de los datos}
\begin{framed}
\textcolor{red}{Esto va acá o es parte de Explotación?}
\end{framed}


Los datos recolectados fueron generados por distintos usuarios, en múltiples sitios y bajo distintas ontologías y estándares.
Estos aún curados, necesitan un último paso para poder realizar una explotación, y es la integración.

Este gran y dificultoso proceso abarca cualquier operación que intente dar una visión más unificada de la información. Este
proceso puede tener distintos niveles y aspectos a integrar, como podría ser por ejemplo, unificar las ontologías de los reviews, o bien
en una nueva ontología de review, o bien en una ya existente, para lograr tener información semánticamente más parecida. Ya 
que este proceso puede ser muy dificultoso, habrá que ver en base a los requerimientos, qué aspectos de los datos se integrarán.

Una vez realizada la integración mínima necesaria par also requerimientos, el dataset ya estará listo para su explotación.

\section{Publicación del Dataset Curado}
???

\section{Explotación del Dataset}
\begin{framed}
\textcolor{red}{Acá es donde efectivamente se discute que implica cubrir los requerimientos de la aplicación . Puede ser que acá tambien te refieras a como la forma en la que querés explotar los datos impacta en las fases anteriores. Incluso, habría que pensarlo, en este capitulo tal vez conviene primero hablar de explotación y luego de las otras fases - porque explotación determina mucho que es lo que vas a mirar en las fases anteriores, ¿no?}
\end{framed}



\chapter{Selección de Vocabularios}
\label{chapter:seleccion}

En la actualidad existen cuatro vocabularios que cumplen los requerimientos mínimos para modelar el dominio de problema planteado.

\subsection{Review Ontology}

RDF Review Vocabulary

También conocido como Review Ontology, es una de las ontologías más antiguas de review, que fue pensada para uso del lenguaje 

RDF y está definido bajo el namespace http://purl.org/stuff/rev\# .

Fue utilizado para la construcción Revyu , y sirvió como guía para otras ontologías.

Consta de tres clases y trece propiedades.

Clases

Comment: Un comentario sobre el review. 

Feedback: Expresa la utilidad del review. 


Review: El review mismo. 


Propiedades


commenter: Especifica el usuario que realizó el comentario del review. Tiene dominio Feedback o Comment y rango foaf:Agent.


Actualmente se encuentra en desuso.


hasReview: Enlaza el ítem evaluado con el review. Su dominio es rdfs:Resource (siendo ésta la clase del ítem evaluado) y su rango es Review. Es una de 
las propiedades principales. 


hasComment: Idem anterior pero con el comentario en lugar del review.

Se encuentra actualmente en desuso.

hasFeedback: Idem anterior pero con feedback en lugar de comentario.

Se encuentra actualmente en desuso.


maxRating: Establece el puntaje máximo que es posible otorgar por un usuario a travez de la propiedad rating. Tiene como dominio 
Review y como rango Literal siendo este último un número positivo. Su ausencia en un review asume su valor por defecto (5). Por lo que 
si bien no es indispensable que esté, se deberá respetar la convención ala hora de generar el rating.



minRating: De la misma forma que el anterior, sólo que establece el puntaje mínimo y su valor por defecto es (1).


positiveVotes: Se refiere a la cantidad de votos positivos que tuvo el review, otorgados por usuarios que lo leyeron y lo encontraron 
útil. SU domino es Review y su rango Literal siendo este último un número positivo.


Se encuentra actualmente en desuso.


rating: Una de las propiedades principales, indica el valor numérico otorgado por el creador del review sobre el ítem evaluado. 
Su domino es Review y su rango es Literal siendo este último un número entre los valores de minRating y maxRating.


reviewer: Especifica el usuario que realizó el review. Tiene dominio Review y rango foaf:Person. 



text: Otra de las propiedades principales, define el texto que describe el sentimento del usuario hacia el ítem. Tiene como 
domino Review y como rango Literal.


totalVotes: Exactamente igual a positiveVotes.


title: El título del review . Tiene dominio Review y rango Literal. Subclase de dc:title


No tiene demasiada utilidad para el caso de estudio y además se encuentra en desuso.


type: Enuncia el tipo de ítem que clasifica taxonómicamente al ítem evaluado. Su domino es rdfs:Resource (siendo ésta la clase del ítem evaluado)  
y su rango no se encuentra especificado. 


Es una propiedad muy útil pero actualmente se encuentra en desuso.


date: Si bien no está definida dentro del vocabulario, es correcto utilizar http://purl.org/dc/terms/date, implica la fecha en la 
que se realizó el review.


\subsection{hReview}

Como se mencionó anteriormente, está establecido por convención, que los microformatos son una forma de publicar información en la 
web semántica, pero al no disponer de namespaces, no pueden ser representados por ninguna ontología o ningún otro lenguaje de la misma. 


Al no poder utilizar ontologías que modelen reviews surgió la necesidad de crear un estándar específico para embeberlos dentro de HTML 
utilizando microformatos.

Dicho estándar se encuentra actualmente en la versión 0.4 y propone el uso de 10 propiedades, que se supone, deberían ser 
suficientes para cubrir todas las necesidades a la hora de generar un review.

summary (opcional): Puede ser el título o nombre del review, o es posible también hacer una pequeña sinopsis del mismo. Se encuentra en desuso.

type (opcional): Representa el tipo de ítem evaluado, pero se encuentra acotado a alguno de estos  product | business | event | person | place | website | url .


También está en desuso.

item: Esta propiedad enuncia toda la información que se crea necesaria para identificar al ítem, mínimamente los atributos nombre, url y foto.


Para lograr bajo una propiedad cubrir todos los atributos, el rango de la misma debería ser un hCard, que luego contendrá las propiedades mínimas necesarias: 
fn, url, photo y cualquier otra que se quiera adicional.

reviewer (opcional): Al igual que ítem, indica todo lo necesario para identificar a la persona autora del review, para lo cual también deberá representarse 
con un hCard. 

dtreviewed (opcional): Se refiere a la fecha en la que fue creado el review. 

rating: El valor numérico con el cual el usuario expresa su satisfacción con el ítem, y está formado por un entero con un solo decimal 
de precisión, que se encuentra dentro del rango 1.0 a 5.0. Dicho rango puede ser alterado con la presencia de las propiedades worst y best 
que restringen el valor mínimo y máximo respectivamente.

descripción (opcional): Establece el valor textual con el cual el usuario expresa su satisfacción con el ítem, creando una 
sinopsis detallada del mismo. 

tags (opcional): Una etiqueta intenta establecer una idea acerque de qué se trata el contenido en una sola palabra para una rápida identificación 
o para mejorar las búsquedas. 

permalink (opcional): Genera una URI que identificará al review creándole una especie de ID, que será útil para los casos donde 
se podría repetir la publicación del mismo. 

license (opcional): Expresa la licencia del review.

El indicador ``(opcional)'' se refiere a si es indispensable para conformar un hReview o no, de manera tal que si no se encuentra una 
propiedad que no está marcada como opcional no podrá ser considerado un hReview.

Cabe destacar que muchas de las propiedades opcionales, podrían ser necesarias para el caso de estudio.

El problema con este vocabulario surge a la hora de trabajar con la información obtenida, que no puede ser representada por ningún otro 
lenguaje de la web semántica, por lo que se vió la necesidad de mapear las propiedades de hReview, a otro vocabulario que sí pueda.

Esto llevó a que en la web donde definen hReview lo consideren compatible con RDF Review Vocabulary, por lo que en Noviembre de 2007 
se creó una herramienta que transforma de uno al otro hreview2rdfxml.xsl,  pero existe un problema con el rango de algunas 
propiedades, por ejemplo reviewer (propiedad homónima en ambos vocabularios, pero una con rango hCard y otra con rango foaf:Person).

En general si bien este vocabulario bien utilizado puede ser efectivo, resulta poco flexible y complciado de manejar por parte 
de quien quiera explotarlos, el motivo por el cual se ha vuelto muy popular es su facilidad para generarlos, dado que 
microformatos es un lenguaje muy sencillo y cualquier persona con un relativamente mínimo ocnocimiento de HTML puede generar 
sin problemas un hReview, teniendo además herramientas online como opción, que generan el código a travez de un formulario. 
Como es el caso de http://microformats.org/code/hreview/creator. 


\subsection{dataVocabulary}
RDF Data Vocabulary

En Mayo de 2009 Google anuncia la introdución de los llamados ``Google Rich Snippets'', estos fragmentos enriquecidos son una convención 
de etiquetas (con soporte para RDFa-lite y microdata) que permitían agregar información útil a los SERP del buscador de Google. De manera tal 
que los datos que contenían estos fragmentos, recibían un tratamento especial.

Sin embargo en el anuncio, Google revela que el soporte se limitó al uso de las clases y propiedades del vocabulario definido en una página 
notoriamente improvisada llamada http://rdf.datavocabulary.org/ . 

En ella se establecían modelos de clases para varios tipos de ítems, tales como Persona, Organización o Producto y también para los Reviews.

La clase Review, quedó definida bajo el namespace http://data-vocabulary.org/Review e incluía las siguientes propiedades:

itemreviewed: Enlace al ítem que está siendo evaluado.

rating: El valor numérico con el cual el usuario expresa su satisfacción con el ítem, tiene como rango un valor numérico bajo la clase 
xsd:string o Rating, y los valores posibles se encuentran en escala de 1 a 5, pudiendo la misma ser alterada con la presencia de las 
propiedades worst y best que restringen el valor mínimo y máximo respectivamente.

reviewer: El autor del review, su rango es dvocab:Person o xsd:string.

dtreviewed: La fecha en la que se realizó el review, no contiene un rango específico pero aclara que debe respetar el formato 
ISO para las fechas.

description: El cuerpo del review que representa el valor textual de satisfacción del usuario con el ítem.

summary: Un resumen corto del review.

Se puede notar la excesiva similitud de este vocabulario con hReview, queda claro que no hubo una intención de innovar algo, 
sino de representar el hReview en microdatos, probablemente por el apuro en la que data vocabulary fue creado.

Vale aclarar que limitar las clases y propiedades posibles en RDFa es básicamente hacerle perder el sentido al lenguaje 
(la decentralización de los vocabularios sobre los términos) haciendo que el lenguaje se utilice como si fuese microformatos, 
pero perdiendo su valor más improtante (la simplicidad) de manera tal que tomó la inflexibilidad de microformatos y la complejidad 
de RDFa.

Más adelante los Google Rich Snippets incluyeron también soporte para microformatos (lo que incluía hReview). 


\subsection{Schema.org}
Schema.org

Luego de los fragmentos enriquecidos y el improvisado vocabulario data-vocabulary creado para soportar los fragmentos, Google 
en conjunto con Bing y Yahoo crearon en el 2011 Schema.org, con el objetivo de obtener un vocabulario más completo y 
organizado para la implementación de los snippets con RDFa y microdata.

Su lanzamiento provocó la inmediata obsoletización de data-vocabulary cuyas clases fueron todas reemplazadas por equivalentes 
dentro de la nueva ontología:

http://www.data-vocabulary.org/Address -> http://schema.org/PostalAddress

http://www.data-vocabulary.org/Geo -> http://schema.org/GeoCoordinates

http://www.data-vocabulary.org/Organization -> http://schema.org/Organization

http://www.data-vocabulary.org/Person -> http://schema.org/Person

http://www.data-vocabulary.org/Event -> http://schema.org/Event

http://www.data-vocabulary.org/Product -> http://schema.org/Product

http://www.data-vocabulary.org/Review -> http://schema.org/Review

http://www.data-vocabulary.org/Offer -> http://schema.org/Offer

Este nuevo vocabulario, recibe constantes actualizaciones, al punto que, al día de la fecha, la ontología cuenta con 946 clases. 

En particular, la clase review, definida bajo el namespace http://schema.org/Review y es subclase de schema:CreativeWork 
que a su vez es subclase de schema:Thing .

Propiedades de Review

itemReviewed	El ítem que está siendo evaluado, tiene rango schema:Thing 

reviewBody 	El valor textual del review de la evaluación, que tiene rango schema:Text 

reviewRating 	El valor numérico del review de la evaluación, que tiene rango schema:Rating 


Propiedades de CreativeWork 

about 	El tema del contenido. Rango schema:Thing. 

aggregateRating 	El promedio de la acumulación de uno o más ratings dentro de un review. Tiene rango schema:AggregateRating 

author 	El autor del contenido. Tiene rango schema:Person o schema:Organization

comment 	Comentarios sobre el contenido. De rango schema:Comment o schema:UserComments.

commentCount 	Cantidad de comentarios. Rango schema:Integer . 

creator 	El autor del contenido. Tiene rango schema:Person o schema:Organization

dateCreated 	Fecha en la que fue creado. Rango schema:Date .

dateModified 	Fecha en la que fue modificado por última vez. Rango schema:Date .

datePublished 	Fecha en de la primer publicación. Rango schema:Date .

publisher 	El publicador. Tiene rango schema:Organization

review 	 	El review sobre el contenido. Tiene rango schema:Review .

text 		Contenido textual. Tiene rango schema:Text .


Properties de Thing

additionalType 	 Tipos adicionales para el ítem que en general son utilizados para especificar tipos externos, como podría ser por ejemplo una película de clase http://schema.org/Movie con el tipo adicional  http://dbpedia.org/ontology/ . Tiene rango schema:URL .

alternateName 	 Especifica un alias. Rango schema:Text .

description 	Una descripción corta del ítem. Rango schema:Text 

name 		Establece el nombre. Rango schema:Text .

sameAs 		Una referencia a una url que desambiguadamente represente al ítem, como podría ser una URL de wikipedia, dbpedia, freebase, etc . Rango schema:URL.

url 		URL del ítem. Rango schema:URL

Unas 60 propiedades de CreativeWork fueron omitidas por no iban al caso ya que sólo tienen razón de ser para otras clases que 
también heredan de CreativeWork. Lo importante aquí es remarcar un par de situaciones interesantes:

=Las propiedades Review:reviewBody, CreativeWork:text y Thing:description podrían contener el valor textual de la evaluación del review semánticamente correcta por la definición de cada una.

=Las propiedades CreativeWork:author y CreativeWork:creator son exactamente iguales. 

=Los valores CreativeWork:dateCreated CreativeWork:dateModified y CreativeWork:datePublished podrían generar confusión. 

=Sería sintácticamente correcto que un Review utilice la propiedad Review .

Más adelante se mostrarán y enumerarán los problemas que estas cuestiones (causadas por el afán de hacer uso de la herencia
lo más posible)  generan. 


\chapter{Recolección de los datos}

 

Objetivo 

Como se indicó anteriormente la información semántica que va a ser necesaria para construir se encuentra en la web en forma de documentos 
HTML que tienen la particularidad de ser muy efímeros, de manera tal que un documento que poseía datos relevantes a la fecha,
puede al día siguiente, o dejar de estar disponible on-line, o haber cambiado de forma tal que la información de éste ya no es 
relevante, o ya no la posee. 

En [] sección 4.2 Challenges for the selection of data sources se generó una estadística de este caso, donde se estableció que 
en promedio 62\% de los documentos entoncontrados, continuaban on-line luego de un año, y de estos, sólo un 56\% aún poseían 
datos relevantes. 

Si bien armar un dataset con sólo información extraída de los documentos sin descargar estos últimos es posible, la situación anterior 
genera la necesidad de mantenerlos en una copia local para evitar una posible pérdida de los mismos. 

El objetivo entonces será armar un repositorio local con los documentos on-line descargados que se cree que tienen la información 
necesaria. 

 

Estrategia 






\include{extraccion}

\chapter{Extracción y Almacenamiento de los Datos}
\label{chapter:extraccion}

Objetivo

Se realizará un proceso de extracción que convierta los documentos con información semántica embebida en documentos HTML en documentos RDF, que luego puedan ser 
almacenados en un triplestore.

La necesidad de tener la información en un triplestore surge de varios puntos:

Tener la información centralizada, así, por cada paso siguiente a realizar, resulte mucho más simple aplicar un mismo proceso a todos los datos.

Tener la información en un mismo lenguaje, por la misma razón que el punto anterior.

Poder realizar queries SPARQL tanto para generar estadísticas como para realizar un curado de la información.

Poder utilizar el motor de inferencias para detectar errores en los documentos.

Estrategia

Se utilizó any23 para extraer el contenido semántico  y generar por cada documento html, un documento en N-Quads.
El motivo de utilizar any23 es que es la única librería que puede ser utilziada en java, y además soporta todos los lenguajes de RDF embebido en HTML.


Luego se utilizó la herramienta RIOT para hacer un merge de todos los documentos nq y generar uno único.

Y po rúltimo se utilizó la herramienta bulkloader para crear una base de datos TDB a partir del archivo nq que contenía todos los documentos nq.


\chapter{Evaluación de Calidad de los Datos}
\label{chapter:evaluacion}

\chapter{Curado de los Datos}
\label{chapter:curado}

Objetivo:
Corregir todos los problemas viables encontrados en el paso anterior que se pueda.

Curado Evaluación nº 1 - Vocabularios:

Estrategia:
Se optó por adicionar al review la propiedad en forma correcta con el mismo recurso destino sin eliminar la propiedad incorrecta. 
Todo se realizó con una simple consulta SPARQL

INSERT{\\
  GRAPH ?g {\\
    ?s ?p ?o .\\
  }\\
}\\
WHERE{\\
  SELECT ?g ?s (IRI(CONCAT("http://schema.org/", ?prop)) AS ?p) ?o\\
  WHERE{\\
    GRAPH ?g{\\
      ?s ?t ?o .\\
      FILTER(REGEX(str(?t), "http://schema.org/*+", "i")).\\
      BIND(REPLACE(str(?t), '^.*(#|/)', "") AS ?prop)\\
    }\\
  }\\
}\\

Resultados:
Se crearon 2498221 propiedades nuevas de forma correcta correspondientes a cada una de las propiedades incorrectas de la ontología 
schema.

Curado Evaluación nº 2 - Duplicados

Estrategia: 

Se recorrieron todos los grupos de duplicados y se procedió a realizar la siguiente operación:
Se le asignó a cada grupo un id único, al cual llamaremos groupId.
Lueog por cada grupo de duplicados se seleccionó de manera arbitraria un representante con un criterio que se describirá a continuación:
Si dentro del conjunto de documentos entre los cuales se encuentran todos los integrantes del grupo de duplicados no existe ninguno 
que contenga un recurso ya elegido anteriormente como representante, se selecciona cualquier recurso de la lista de duplicados al azar.
Caso contrario, se selecciona el primer recurso encontrado en la lista, que pertenezca al documento que ya contiene algún recurso escogido 
como representante.
Esto se realizó de esa forma, para que todos los representantes se concentren en los mismos documentos y no suceda que dados 2 
documentos idénticos con 2 reviews cada uno, no terminen con un representante cada uno. De esta forma se minimizan la cantidad de 
documentos con reviews en el dataset.
Luego de elegir un representante para el grupo de duplicados, a ese recurso se le asignó una propiedad http://local.org/representantOf y 
como objeto de la propiedad el groupId.
Y para todos los recursos restantes del grupo de duplicados, se les asignó una propiedad http://local.org/duplicatedOf con objeto 
el groupId.

Resultados:

Se asignaron 78705 propiedades duplicatedOf y 17343 representantOf

Curado Evaluación nº 3 - RDFUnit Automático:

Muchos de los problemas detectados por el framework, no otorgaban demasiada información, como por ejemplo
http://schema.org/datePublished does not have datatype: http://www.w3.org/2001/XMLSchema\#date el cual sólo especifica que le 
falta el tipo al recurso objeto, pero no necesariamente significa que esté mal, esto da un indicio de que algo puede andar mal
y será verificado en la consulta manual, en este caso por ejemplo, lo importante es que la fecha tenga el formato apropiado.
Se decidió entonces atacar los problemas más específicos:

http://schema.org/ratingValue has rdfs:domain different from: http://schema.org/Rating

Estrategia:
Primero hubo que investigar, cuáles son todos los dominios para esa propiedad que aparecen en el dataset, para ver si se pueden 
acomodar. Esto se realizó con la siguiente consulta sparql:

select distinct ?dominio (count (?s) as ?cantidad) \\
where{\\
?s <http://schema.org/ratingValue> ?value .\\
?s a ?dominio .\\
}GROUP BY ?dominio\\

Y los resultados fueron

\begin{tabular}{| l | c |}
 <http://schema.org/Rating> & 120165\\
 <http://schema.org/AggregateRating> & 54815 \\
 <http://schema.org/Review> & 5867 \\
 <http://schema.org/Product> & 331 \\
\end{tabular}

Se puede apreciar que en la mayor parte de los casos de error, se dieron porque se incluyó la propiedad ratingValue directamente en 
el Review. Ahora surge la necesidad de saber si esos casos donde el Review tiene la propiedad ratingValue, existe una propiedad reviewRating 
con su respectivo Rating.
Se relizó la consulta:
select (count (?s) as ?cantidad) \\
where{\\
?s a <http://schema.org/Review> . \\
?s <http://schema.org/ratingValue> ?value . \\
?s <http:schema.org/reviewRating> ?rating . \\
}

El resultado fue 0.

Por lo tanto ese problema se solucionó con la siguiente consulta:

DELETE { \\
GRAPH ?g{ \\
?s <http://schema.org/ratingValue> ?value . \\
} \\
} \\
INSERT{ \\
GRAPH ?g{ \\
?s <http://schema.org/reviewRating> ?rating . \\
?rating a <http://schema.org/Rating> . \\
?rating <http://schema.org/ratingValue> ?value . \\
}\\
}\\
WHERE{\\
GRAPH ?g {\\
?s a <http://schema.org/Review> .\\
?s <http://schema.org/ratingValue> ?value .\\
}\\
}\\

schema:worstRating and schema:bestRating has rdfs:domain different from: http://schema.org/Rating
Estrategia:
El proceso es igual al anterior y los dominios disintos de rating eran Review, la misma cantidad que en el anterior 5867

Resultados: 
Idem los anteriores.

http://schema.org/itemReviewed is missing proper range

Estrategia:
Esta evaluación se refiere a que existen recursos objeto de dicha propeidad, que no tienen un tipo específico. Para ver si 
puede ser resuelto, primer hubo que evaluar qué clase de recursos contienen como objeto, esas propiedades que presentaron problemas.

Resultado:
La evaluación encontró que, todos esos recursos eran URIs de un dominio web 411ca.com para ser derreferenciados. La solución podría haber sido 
descargar dichos recursos y anexarlos a la propiedad, pero lamentablemente dichos recursos no se encontraban disponibles de forma online.

Curado Evaluación nº 3 - RDFUnit Manual:

\{DateProperty} has a format different to YYYY-MM-DD:
Para todas las propiedades que indiquen una fecha, los pasos para reparar este caso son los mismos.

Cuando nos encontramos con este caso, se requiere de otra evaluación para entender cómo encarar el problema. Es una evaluación cuyo resultado
pueda responder la siguiente pregunta:

Si no tiene formato YYYY-MM-DD, ¿Qué formato tiene?

La idea en esta evaluación entonces, es encontrar qué formatos distintos contienen los datos, para poder trasladarlos al correcto.

Estrategia:

Comenzando con la siguiente consulta:

SELECT distinct ?date 
WHERE{
?review <http://local.org/id> ?id .
?review {DateProperty} ?date .
FILTER (!REGEX(str(?date), "^[0-9]{4}-[0-9]{2}-[0-9]{2}\$", "i")) .
FILTER NOT EXISTS{
?review <http://local.org/deplicateOf> ?dup.
}
}

Esta consulta retorna la fecha de todos los reviews que no estén duplicados y que además dicha fecha no tenga el formato YYYY-MM-DD

Luego observando los resultados manualmente, se prsta atención also primeros resultados y se trata de encontrar con qué patrón están formados 
para luego anotarlo y realizar la misma consulta esta vez filtrando los reusltados ocn dicho patrón. Iterando este proceso hasta que 
queden 0 resultados.

Resultados:

Se descubrieron los siguientes patrones
\\
\\
http://schema.org/datePublished
\begin{tabular}{| l | c |}
Expresión regular & Cantidad de reviews\\
^[A-z]+ [0-9]{2}, [0-9]{4}\$ & 14715 \\
^[0-9]{4}-[0-9]{2}-[0-9]{2}.+ & 17700 \\
^[0-9]{4}-[0-9]{2}-[0-9]{1} [0-9]{2}:[0-9]{2}:[0-9]{2}\$ & 7179 \\
^[0-9]{2}\\.[0-9]{2}\\.[0-9]{4}\$ & 7172\\
^[A-z]+ [0-9]{2}, [0-9]{4}.+ & 5049\\
^[A-z]{3} [0-9]{1}, [0-9]{4} & 1847\\
^[0-9]{2}\\.[0-9]{1}\\.[0-9]{4}\$ & 410\\
^[0-9]{1}\\.[0-9]{1}\\.[0-9]{4}\$ & 120\\
[0-9]{2}/[0-9]{2}/[0-9]{2} & 67\\
^[0-9]{1}\\.[0-9]{2}\\.[0-9]{4}\$ & 30
\end{tabular}

http://purl.org/dc/terms/date
\begin{tabular}{| l | c |}
Expresión regular & Cantidad de reviews\\
^[0-9]{4}-[0-9]{2}-[0-9]{2}.+ & 33833\\
^[A-z]+.[A-z]{2}.[0-9]{4}-[0-9]{2}-[0-9]{2}\$ & 33058\\
^[A-z]{3}\\n.*[0-9]{1}\\n.*[0-9]{4}\$ & 25543\\
^[A-z]{3}\\n.*[0-9]{2}\\n.*[0-9]{4}\$ & 17770\\
^[A-z]+ [0-9]{2}, [0-9]{4}\$ & 17069\\
[0-9]{2}((?!-).)[0-9]{2}((?!-).)[0-9]{4} & 11720\\
^[0-9]{2} [A-zéû]+ [0-9]{4} & 4147\\
^[0-9]{2}((?!-).)[0-9]{2}((?!-).)[0-9]{4} & 4093\\
^[0-9]{2}((?!-).)[0-9]{1}((?!-).)[0-9]{4} & 2803\\
^[A-z]{3} [0-9]{1}, [0-9]{4} & 2783\\
^[A-z]+ [0-9]{2}[a-z]{2} [0-9]{4}\\.\$ & 2058\\
^[0-9]{2} [A-z]{3} [0-9]{4} & 1730\\
^[0-9]{1}((?!-).)[0-9]{1}((?!-).)[0-9]{4}\$. & 1103\\
^[0-9]{2}-[0-9]{2}-[0-9]{4}\$ & 1024\\
^[A-z]+ [0-9]{1}[a-z]{2} [0-9]{4}\\.\$ & 746\\
^[0-9]{1}((?!-).)[0-9]{2}((?!-).)[0-9]{4}\$ & 279\\
\end{tabular}

\\
http://schema.org/publishDate
\begin{tabular}{| l | c |}
Expresión regular & Cantidad de reviews\\
[0-9]{2}/[0-9]{2}/[0-9]{2} & 2511\\
^[0-9]{2}\\.[0-9]{2}\\.[0-9]{4}\$ & 2207
\end{tabular}

\\
http://schema.org/dtreviewed
\begin{tabular}{| l | c |}
Expresión regular & Cantidad de reviews\\
^[0-9]{1}((?!-).)[0-9]{2}((?!-).)[0-9]{4}\$ & 14762\\
^[0-9]{1}((?!-).)[0-9]{1}((?!-).)[0-9]{4}\$ & 6358\\
^[0-9]{2}((?!-).)[0-9]{2}((?!-).)[0-9]{4}\$ & 4193\\
^[0-9]{2}((?!-).)[0-9]{1}((?!-).)[0-9]{4}\$ & 1590
\end{tabular}

Una vez encontrados los patrones, sólo fue cuestion de correr un algoritmo en Java dónde se contemplan todo los patrones y se 
le aplica la correción apropiada al String y se vuelve a almacenar en el review.
También en varios casos hubo que verificar dónde se encontraba el día y dónde el mes como por ejemplo para casos como éste ^[0-9]{2}\\.[0-9]{2}\\.[0-9]{4}\$ & 2207 
Manualmente se observan los resultados de ese patrón y se puede observar si los primeros 2 números son mes o día.



\chapter{Integración de los Datos}
\label{chapter:integracion}

Como quedó explicado en la sección estrategia, integración comprende los procedimientos que se realizan con el objetivo de obtener una
visión más unificada del dataset.
Esto puede significar, modificar los datos y ontologías y también agregar información faltante en los ítems/reviews.
Este es el paso más importante de todo el proceso y la posibilidad de conseguir realizar una aplicación correcta que satisfaga los requerimientos
depende del éxito obtenido en el mismo.
En base a los requerimientos establecidos, se pueden pensar en distintos procedimientos que son necesarios para poder explotar los datos
correctamente:
\\
\\
Unificación de vocabularios: En el primer paso del proceso se realizó una selección de los vocabularios con los que se trabajará, 
estos vocabularios modelan el mismo dominio de información pero muchas veces de forma distinta. Explotar información modelada de 
diferentes maneras puede resultar innecesariamente engorroso, y también para quienes en un futuro quieran hacer uso del dataset (ya que
deberán consutlar la información previamente conociendo todas las ontologías de review existentes).\\
Unificación de autores: Es muy importante para lograr cumplir el objetivo de los requerimientos que si dos reviews distintos fueron 
generados por el mismo autor, esto quede explícitamente asentado, para poder realizar los algoritmos de recomendación.\\
Unificación de ítems: El paso más difícil y más importante, si un ítem está modelado dos veces de forma distinta debería saberse que 
se trata del mismo.\\
Unificación de tipos de ítems: Esto es muy útil para darle más posibilidades a la aplicación, y es agrupar los ítems según su tipo 
(Libro, película, Hotel, Auto, etc), de manera que por ejemplo si un usuario quiere que le recomienden un libro, la aplicación sólo 
le ofresca libros. Para la ontología schema, esto puede ser más sensillo dado que los ítems ya se encuentran con su tipo identificado con 
la clase, en el caso de libro, sería cuestión de buscar ítems con la clase schema:Book. El problema es que esto puede no ocurrir siempre, 
Un libro, puede estar clasificado como schema:Product, por lo que necesitaría que también lo clasifiquen como Book.\\
\\
\\
Por razones de que la tesis ya cubrió el desarrollo suficiente, sólo se intentarán implementar los puntos 1 y 4 de la integración. Ya que 
los otros dos son demasiado abarcativos y pueden requerir demasiado trabajo.\\
\\
Unificación de vocabularios\\
\\
Objetivo: Conservar los reviews del dataset en una única ontología. Lo que simplificará las búsquedas, agregaciones y otras operaciones.
Además de contener datos semánticamente más parecidos. Y también facilitará la comprensión del dataset a quienes quieran hacer uso del mismo
en un futuro (ya que sólo deberán concentrarse en una sola ontología)
\\
Estrategia: Deberá como primer paso realizarse una selección de vocabulario, en la cual se escoja uno para que quede como el único en el dataset.
En este caso de estudio el vocabulario elejido es Review Ontology, ya que todas sus propiedades son suficientes para cumplir los requerimientos y 
este es mucho más simple, lo que facilitará su utilización. El vocabulario de schema poseé demasiadas propiedades que son irrelevantes para este caso de estudio. 
Además vale la pena recordar los problemas encontrados en el capítulo evaluación causados por la ontología.
Como segundo paso, buscar para cada propiedad de schema:Review utilizada en el dataset su contraparte en la ontología purl.\\
La equivalencia entre las propiedades ya fueron realizadas en el capítulo de selección de vocabularios. Pero sólo fue realizado con las propiedades 
relevantes al caso de estudio. Las siguientes propiedades (que valga la redundancia son irrelevantes al caso de estudio) fueron utilizadas dentro del dataset, y 
no se estableció su equivalente en el otro vocabulario:
http://schema.org/about y http://schema.org/keywords. 
Si bien, la falta de estas propiedades no impactará sobre la aplicación final, es conveniente no perder datos para la publicación del dataset, ya que 
a alguien podría resultarle útil. 
En la etapa de evaluación se observó que la propiedad about, se utilizó en el dataset sólo para poner contener el valor ``This is required'' en todos los casos. 
Por lo que no incluir esta propiedad en el dataset no producirá una pérdida de información.
Y la propiedad keywords puede ser reemplazada por http://purl.org/dc/terms/subject. Ya que en la documentación de dublin core, proponen la utilización de esta propiedad 
utilizando keywords.
El caso de reviewRating, que tiene como objeto un schema:Rating, no puede ser explicitamente reemplazado, ya que esa propiedad 
sólo es utilizada para desacoplar el rating del review, situación que no se da en la otra ontología. Pero las propiedades dentro del rating se encuentran todas 
con su equivalente en purl, por lo tanto no causará ningún problema.
Por último, schema:Person que es rango de la propiedad author, debería ser mapeado a foaf:Person que es objeto d ela propiedad reviewer. 
El problema se encuentra en lo analizado en la etapa de curado, que se observó que los nombres de los autores podía ser nomrbes completos o usernames. 
Y no existe una propiedad dentro del vocabulario de foaf:Person que sea tan genérico como para contemplar todo slos casos posibles, como sí existe en vCard.
Además de que todos los previos valores de reviewer están conformados por un vCard, como también ya se analizó en la etapa de curado.
Por estos motivos se determinó continuar utilizando vCards en lugar de recursos de tipo foaf, y dichos vCards utilizarán la propiedad
fn (que contempla todos los casos mencionados).
Finalmente el mapeo de propiedades quedaría:\\
\begin{tabular}{| l | c |}
Propiedad schema & Propiedad de reemplazo\\
name & title \\
reviewBody & text \\
datePublished & dc:date \\
author & reviewer \\
ratingValue & rating\\
bestRating (reviewRating) & maxRating\\
worstRating (reviewRating)& minRating\\
url & vcard:url\\
keywords & dc:subject\\
name (author) & vcard:fn (reviewer)\\
comments & hasComment\\
author (comment) & commenter \\
text (comment) & commentText 
\end{tabular}

	
\chapter{Publicación del Dataset Curado}
\label{publicacion}


\chapter{Explotación del Dataset}
\label{chapter:explotacion}

En la Sección \ref{section:explotacion} se habló de la elección de un sistema de recomendación, previo 
a la construcción de la aplicación con el objetivo de aprovechar lo más posible la información.\\
\\
Hasta aquí se generó un dataset curado, que en la sección de integración por cuestiones de complejidad 
(e incluso de posibilidad) no se realizó la integración tanto de los autores, como de los ítems.

Dado que no se dispone de ninguna propiedad para determinar que tanto dos autores son el mismo o dos ítems son el mismo 
resulta inadecuado la utilización de un sistema de recomendación de filtrado colaborativo.

Esto también puede incluir a los sistemas de recomendación basados en contenido, porque también hacen uso de 
un perfil de usuario que requiere mucha cantidad de información de cada uno para su construcción.
\\\\
Es por eso que sólo resta la implementación de un sistema de recomendación no personalizado.

De manera que los rankings generados por la aplicación, van a ser idénticos para cualquier 
usuario que lo solicite.
\\\\
Se generó entonces una aplicación que hace uso del dataset publicado en el capítulo anterior, 
cumpliendo con los requerimientos planteados en la Sección \ref{section:caso-de-estudio}.

La figura \ref{figure:explotacion} muestra un diagrama de la arquitectura de la aplicación compuesta por tres capas:

\begin{figure}
    \centering
    \includegraphics[width=0.8\textwidth,natwidth=610,natheight=642]{explotacion}
    \caption{Arquitectura de la aplicación}
    \label{figure:explotacion}
\end{figure}

\begin{enumerate}
 \item API Restful: Que provee una API mediante la utilización del framework Jersey. Esto evitó 
 que se requiera utilizar una interfaz gráfica que no cumple ningún aporte a los objetivos de esta tesis.
 \item Service: Que proporciona la funcionalidad de la aplicación, generando la estructura de los 
 rankings en el sistema de recomendación, para luego utilizar el framework Jena que modela y proveé la información 
 solicitada por el sistema. 
 
 También posee una librería que mapea la información otorgada pro Jena a Strings JSON que pueden ser 
 devueltos por la API.
 \item Queries: Por último se encuentra la capa queries, que se encarga de obtener los datos del dataset 
 mediante consultas SPARQL generadas por la librería ARQ del framework Jena. Dicha capa será la encargada 
 de interactuar con el SPARQL Endpoint disponible por el servidor Fuseki.
\end{enumerate}

\chapter{Conclusiones y trabajo futuro}
\label{chapter:conclusiones}

Se demostró cómo puede hacerse uso de las tecnologías, frameworks y estándares de la Web 
Semántica para explotar las contribuciones de los usuarios a lo largo de toda la web dentro 
del marco de los Sistemas de Recomendaciones.
\\\\
A partir de todo el proceso se pueden distinguir dos actores que participan en la publicación 
de reviews en la web semántica, el usuario que genera la información y el desarrollador
del sitio que proveé al usuario la plataforma para hacerlo.
\\\\
La importancia entonces de la existencia de vocabularios expresivos, concretos y precisos 
resulta fundamental para ayudar a estos dos a publicar la información correctamente, ya que 
se demostró que a partir del paso Evaluación, la incorrectitud en la información publicada 
crea grandes problemas que resultan muy complicados de resolver, al igual que se demostró 
que muchos de los errores son muy influenciados por los vocabularios.
\\\\
Aún con vocabularios completos, correctos y precisos, continuarían presentándose dificultades en 
el paso de Integración, ya que es muy dependiente de la relación entre los datos publicados 
por distintos usuarios.

Intentar integrar ítems y autores parece ser entonces una tarea que va de extremadamente compleja 
a imposible, dado que la mayoría de los reviews poseén de información de los autores sólo un username, 
en lugar de un identificador global como podría ser una dirección de email.

También ítems, que tampoco poseén identificadores globales, sino nombres escritos en 
distintos formatos.

Esto parece tener origen en la naturaleza decentralizada de la web semántica, que otorga 
a los usuarios la libertad de generar recursos sin límite.

La solución no parece encontrarse por el camino de la centralización, como fue planteado en \cite{Heath2006J} cuya
aplicación resultante terminó siendo obsoleta,
sino con la estandarización más restrictiva en vocabularios y propiedades, intentando 
llevar a los usuarios a la utilización de esos tan necesarios identificadores.
\\\\
Por otro lado se pudo observar la preferencia de los desarrolladores de sitios a la utilización de 
formatos de código semántico embebido en HTML, en lugar de documentos RDF puros. Aunque cada uno de 
los formatos poseé ventajas y desventajas para dicho desarrollador del sitio, se notó que uno de ellos 
(microformatos) sólo contnía desventajas desde el punto de vista del desarrollador de la aplicación 
semántica. 

Dicha conclusión  se obtiene de verificar que la calidad de los ítems escritos en microformatos del 
dataset es muy baja, ya que sólo poseén como información un nombre en la mayoría de sus casos.

Esto se debe a que no se disponen de namespaces para la utilización de ontologías que permitan 
la utilización de propiedades descriptivas necesarias, y sólo cuenten con el uso de V-Cards.

El punto de vista del desarrollador de aplciaciones semánticas por supuesto que no suele ser tenido 
en cuenta por el desarrollador de sitios, que la mayoría de las veces opta por el uso de 
microformatos por su simplicidad.
\\\\
A pesar de los problemas mencionados, se alcanzó el objetivo de construir un sistema de recomendación, 
dondé se produjeron satisfactorios resultados en los puntos de evaluación y curación. 
Y dicha construcción demostró cómo puede ser necesarios combinar vocabularios (como se vió en 
el capítulo \ref{chapter:integracion} que se aprovechó la simplicidad y correctitud de la 
Review Ontology para los reviews para los ítems de schema), y mostró interesantes usos 
de la minería de datos para el proceso.

También mostró la necesidad de conocer las distintas herramientas y aplicaciones como Sindice 
o RDFUnit, que aportan mucho al proceso, también de la necesidad de conocer la naturaleza de 
los datos con los que se trabaja.
\\\\
Quedan entonces planteados, desafíos en varios de los puntos de este trabajo, comenzando con 
la proposición de estándares o soluciones que resuelvan el problema de la publicación de información 
no identificable, siguiendo por el análisis del uso de la minería de datos en varios puntos, y 
la generación de una estrategia de integración exitosa en este contexto. 

Otro aspecto a investigar es el de la obtención de fuentes de datos, más completa y actual que
no necesite de la utilización de complejos recur5sos de hardware para un crawling, dado 
que Sindice a partir de 2015 ha dejado de funcionar como web crawler.

Un punto interesante de partida podría ser la inspección del nuevo LOD Laudromatic.

Por último también queda por mejorar los puntos de Evaluación y Curación, que pueden analizarse 
cuánto más completo y eficiente pueden lograrse.


%la bilbiografia se pone en el archivo bibliografia.bib en un formato que se llama bibtex - luego lo vemos juntos. Puse un ejemplo. 
\chapter{Bibliografía}
\bibliographystyle{plain}
\bibliography{bibliografia}

\appendix
\label{anexo}

\chapter{Publicaciones}



\end{document}
