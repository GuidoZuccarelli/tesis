\chapter{Integración de los Datos}
\label{chapter:integracion}

Como quedó explicado en la sección estrategia, integración comprende los procedimientos que se realizan con el objetivo de obtener una
visión más unificada del dataset.
Esto puede significar, modificar los datos y ontologías y también agregar información faltante en los ítems/reviews.
Este es el paso más importante de todo el proceso y la posibilidad de conseguir realizar una aplicación correcta que satisfaga los requerimientos
depende del éxito obtenido en el mismo.
En base a los requerimientos establecidos, se pueden pensar en distintos procedimientos que son necesarios para poder explotar los datos
correctamente:
\\
\\
Unificación de vocabularios: En el primer paso del proceso se realizó una selección de los vocabularios con los que se trabajará, 
estos vocabularios modelan el mismo dominio de información pero muchas veces de forma distinta. Explotar información modelada de 
diferentes maneras puede resultar innecesariamente engorroso, y también para quienes en un futuro quieran hacer uso del dataset (ya que
deberán consutlar la información previamente conociendo todas las ontologías de review existentes).\\
Unificación de autores: Es muy importante para lograr cumplir el objetivo de los requerimientos que si dos reviews distintos fueron 
generados por el mismo autor, esto quede explícitamente asentado, para poder realizar los algoritmos de recomendación.\\
Unificación de ítems: El paso más difícil y más importante, si un ítem está modelado dos veces de forma distinta debería saberse que 
se trata del mismo.\\
Unificación de tipos de ítems: Esto es muy útil para darle más posibilidades a la aplicación, y es agrupar los ítems según su tipo 
(Libro, película, Hotel, Auto, etc), de manera que por ejemplo si un usuario quiere que le recomienden un libro, la aplicación sólo 
le ofresca libros. Para la ontología schema, esto puede ser más sensillo dado que los ítems ya se encuentran con su tipo identificado con 
la clase, en el caso de libro, sería cuestión de buscar ítems con la clase schema:Book. El problema es que esto puede no ocurrir siempre, 
Un libro, puede estar clasificado como schema:Product, por lo que necesitaría que también lo clasifiquen como Book.\\

