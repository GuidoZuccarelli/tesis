hReview\\
Como se mensionó anteriormente, está establecido por convención, que los microformatos son una forma de publicar información en la 
web semántica, pero al no disponer de namespaces, no pueden ser representados por ninguna ontología o ningún otro lenguaje de la misma. \\
Al no poder utilizar ontologías que modelen reviews surgió la necesidad de crear un estándar específico para embeberlos dentro de HTML 
utilizando microformatos.\\
Dicho estándar se encuentra actualmente en la versión 0.4 y propone el uso de 10 propiedades, que se supone, deberían ser 
suficientes para cubrir todas las necesidades a la hora de generar un review.\\
\\
summary (opcional): Puede ser el título o nombre del review, o es posible también hacer una pequeña sinopsis del mismo. Se encuentra en desuso.\\
\\
type (opcional): Representa el tipo de ítem evaluado, pero se encuentra acotado a alguno de estos  product | business | event | person | place | website | url .\\
También está en desuso.\\
\\
item: Esta propiedad enuncia toda la información que se crea necesaria para identificar al ítem, mínimamente los atributos nombre, url y foto.\\
Para lograr bajo una propiedad cubrir todos los atributos, el rango de la misma debería ser un hCard, que luego contendrá las propiedades mínimas necesarias: 
fn, url, photo y cualquier otra que se quiera adicional.\\
\\
reviewer (opcional): Al igual que ítem, indica todo lo necesario para identificar a la persona autora del review, para lo cual también deberá representarse 
con un hCard. \\
\\
dtreviewed (opcional): Se refiere a la fecha en la que fue creado el review. \\
\\
rating: El valor numérico con el cual el usuario expresa su satisfacción con el ítem, y está formado por un entero con un solo decimal 
de precisión, que se encuentra dentro del rango 1.0 a 5.0. Dicho rango puede ser alterado con la presencia de las propiedades worst y best 
que restringen el valor mínimo y máximo respectivamente.\\
\\
descripción (opcional): Establece el valor textual con el cual el usuario expresa su satisfacción con el ítem, creando una 
sinopsis detallada del mismo. \\
\\
tags (opcional): Una etiqueta intenta establecer una idea acerque de qué se trata el contenido en una sola palabra para una rápida identificación 
o para mejorar las búsquedas. \\
\\
permalink (opcional): Genera una URI que identificará al review creándole una especie de ID, que será útil para los casos donde 
se podría repetir la publicación del mismo. \\
\\
license (opcional): Expresa la licencia del review.\\
\\
El indicador ``(opcional)'' se refiere a si es indispensable para conformar un hReview o no, de manera tal que si no se encuentra una 
propiedad que no está marcada como opcional no podrá ser considerado un hReview.\\
Cabe destacar que muchas de las propiedades opcionales, podrían ser necesarias para el caso de estudio.\\
\\
El problema con este vocabulario surge a la hora de trabajar con la información obtenida, que no puede ser representada por ningún otro 
lenguaje de la web semántica, por lo que se vió la necesidad de mapear las propiedades de hReview, a otro vocabulario que sí pueda.\\
Esto llevó a que en la web donde definen hReview lo consideren compatible con RDF Review Vocabulary, por lo que en Noviembre de 2007 
se creó una herramienta que transforma de uno al otro hreview2rdfxml.xsl,  pero existe un problema con el rango de algunas 
propiedades, por ejemplo reviewer (propiedad homónima en ambos vocabularios, pero una con rango hCard y otra con rango foaf:Person).\\
\\
En general si bien este vocabulario bien utilizado puede ser efectivo, resulta poco flexible y complciado de manejar por parte 
de quien quiera explotarlos, el motivo por el cual se ha vuelto muy popular es su facilidad para generarlos, dado que 
microformatos es un lenguaje muy sensillo y cualquier persona con un relativamente mínimo ocnocimiento de HTML puede generar 
sin problemas un hReview, teniendo además herramientas online como opción, que generan el código a travez de un formulario. 
Como es el caso de http://microformats.org/code/hreview/creator. 