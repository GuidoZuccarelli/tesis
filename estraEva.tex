Como sabemos los reviews son creados por usuarios que en su mayoría no están familiarizados con el desarrollo de una 
aplicación, esto causa que una parte muy significativa del contenido publicado no esté de la forma adecuada para ser procesado. 
La falta de calidad en el contenido publicado puede deberse tanto a errores por parte del usuario como por parte del publicador.\\
El publicador deberá asegurarse que los datos respeten rigurosamente las ontologías en las cuales se publican.\\
El sitio web en el cual el usuario se encuentre realizando el review deberá guiarlo en todo lo posible para lograr que la evaluación
quede en un formato adecuado.\\
Aunque también existen muchos problemas que no dependen del sitio web, generalmente errores semánticos de calidad, donde lo 
redactado esta hecho de forma inconsistente o insuficiente.\\
Este paso entonces tiene por objetivo encontrar todos los problemas de calidad que pueda haber en el dataset que acaba de 
ser descargado y extraído y que generen inconvenientes para una posterior integración/explotación.\\